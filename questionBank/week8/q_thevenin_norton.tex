% Author: Yannan Tuo, Sukrit Arora, Nikhil Dilip
% Email: ytuo@berkeley.edu


\qns{Thevenin Norton}
In this question, you will explore Thevenin Norton equivalence. This is a way of represent the input/output characteristics of a complicated circuit by modeling it as a simple voltage/current source and resistor. Suppose you are given a circuit and 2 terminals A and B. The Thevenin equivalent circuit with respect to those terminals is a circuit of the following form:

\begin{center}
\begin{circuitikz}
\draw(0,4)
to[short] ++(0,-1)
to[V_, l=$V_{oc}$] ++(0,-2)
to[short] node[ground] {} ++(0,-2)
to[short] ++(0,1);


\draw(0,4)
to[R = $R_{eq}$, *-o] ++(2, 0)
to node[right] {$A$} ++(0,0);


\draw(0, 0)
to[short, *-o] ++(2,0)
to node[right] {$B$} ++(0,0);
\end{circuitikz}
\end{center}

where $V_{OC}$ and $R_{eq}$ are set so that the open circuit (OC) voltage between A and B, as well as the short circuit (SC) current through A and B, are the same as in the original circuit. And the Norton equivalent circuit is a circuit of the following form, with the same properties:

\begin{center}
\begin{circuitikz}
\draw(0,4)
to[short] ++(0,-1)
to[I, l_=$I_{sc}$, invert] ++(0,-2)
to[short] node[ground] {} ++(0,-2)
to[short] ++(0,1);


\draw(0,4)
to[short, *-o] ++(3, 0)
to node[right] {$A$} ++(0,0);

\draw(2, 4)
to[R = $R_{eq}$] ++ (0, -4);

\draw(0, 0)
to[short, *-o] ++(3,0)
to node[right] {$B$} ++(0,0);
\end{circuitikz}
\end{center}

\begin{enumerate}
\qitem{ 

Consider the following circuit. Find the equivalent resistance across the A and B terminal.

%feel free to modify this
	\begin{center}
	\begin{circuitikz}

	\draw(0,4)
	to[short] ++(0,-1)
	to[V_=$8V$] ++(0,-2)
	to[short] node[ground] {} ++(0,-2)
	to[short] ++(0,1);
	
	
	\draw(0,4)
	to[R, l = $R$] ++(2, 0)
	to[I, l= $2A$] ++(0, -4)
	to[R, l = $R$, i = $i$] ++(-2,0);
	
	\draw(2,4)
	to[short] ++(2, 0)
	to[short, *-o] ++(1,0)
	to node[right] {$A$} ++(0,0);
	
	\draw(4, 0)
	to[short, *-o] ++(1,0)
	to node[above] {$B$} ++(0,0);
	
	\draw(2, 0)
	to[R, l = $R$] ++(2, 0)
	to[R, l = $R$] ++(0,4);
	
	%https://www.allaboutcircuits.com/technical-articles/thevenin-theorem-dependent-source-circuits/
	%\draw(
	%to[R] ++(0,-4)
	%to[R] ++(-2,0)
	%to[R, l=$7k\Omega$, v=$ $, i=$ $] ++(0,4)
	%to[R, l=$6k\Omega$, v=$ $, i=$ $] ++(-2,0);

	%\draw(2,0)
	%to[short] ++(2,0)
	%to[I, l=$3mA$] ++(0,4)
	%to[short] ++(-2,0);
	\end{circuitikz}
	\end{center}

}

\ans{
The first thing we want to do while solving for the equivalences is null out all our sources. \textbf{This is only possible if all the sources in the circuit are independent sources}.\\

Recall that a null-ed voltage source is a short (because it adds no voltage, it is effectively a wire) and a null-ed out current source is an open circuit (because no current can flow).\\

The resulting circuit would be:

	\begin{center}
	\begin{circuitikz}

	\draw(0,4)
	to[short] ++(0,-3)
	to[short] node[ground] {} ++(0,-2)
	to[short] ++(0,1);
	
	
	\draw(0,4)
	to[R, l = $R$] ++(2, 0)
	to[short, *-o] ++(0, -1);
	
	\draw(0, 0)
	to[R, l = $R$] ++(2,0)
	to[short, *-o] ++(0, 1);
	
	\draw(2,4)
	to[short] ++(2, 0)
	to[short, *-o] ++(1,0)
	to node[right] {$A$} ++(0,0);
	
	\draw(4, 0)
	to[short, *-o] ++(1,0)
	to node[right] {$B$} ++(0,0);
	
	\draw(2, 0)
	to[R, l = $R$] ++(2, 0)
	to[R, l = $R$] ++(0,4);
	
	%https://www.allaboutcircuits.com/technical-articles/thevenin-theorem-dependent-source-circuits/
	%https://www.allaboutcircuits.com/technical-articles/thevenin-theorem-dependent-source-circuits/
	%\draw(
	%to[R] ++(0,-4)
	%to[R] ++(-2,0)
	%to[R, l=$7k\Omega$, v=$ $, i=$ $] ++(0,4)
	%to[R, l=$6k\Omega$, v=$ $, i=$ $] ++(-2,0);

	%\draw(2,0)
	%to[short] ++(2,0)
	%to[I, l=$3mA$] ++(0,4)
	%to[short] ++(-2,0);
	\end{circuitikz}
	\end{center}

Now that we have drawn our circuit, we need to trace all possible paths between A and B and calculate our equivalent resistance. \\

We see we have the two paths colored below:\\

    \begin{center}
	\begin{circuitikz}

	\draw[color=brown](0,4)
	to[short] ++(0,-4);
	
	\draw(0, 0)
	to[short] node[ground] {} ++(0,-1);
	
	
	\draw[color=brown](0,4)
	to[R, l = $R$] ++(2, 0);
	
	\draw(2, 4)
	to[short, *-o] ++(0, -1);
	
	\draw[color=brown](0, 0)
	to[R, l = $R$] ++(2,0);
	
	\draw(2,0)
	to[short, *-o] ++(0, 1);
	
	\draw[color=brown](2,4)
	to[short] ++(2, 0);
	
	\draw(4, 4)
	to[short, *-o] ++(1,0)
	to node[right] {$A$} ++(0,0);
	
	\draw(4, 0)
	to[short, *-o] ++(1,0)
	to node[right] {$B$} ++(0,0);
	
	\draw[color=brown](2, 0)
	to[R, l = $R$] ++(2, 0);
	
	\draw[color=red](4, 0)
	to[R, l = $R$] ++(0,4);
	
	%https://www.allaboutcircuits.com/technical-articles/thevenin-theorem-dependent-source-circuits/
	%https://www.allaboutcircuits.com/technical-articles/thevenin-theorem-dependent-source-circuits/
	%\draw(
	%to[R] ++(0,-4)
	%to[R] ++(-2,0)
	%to[R, l=$7k\Omega$, v=$ $, i=$ $] ++(0,4)
	%to[R, l=$6k\Omega$, v=$ $, i=$ $] ++(-2,0);

	%\draw(2,0)
	%to[short] ++(2,0)
	%to[I, l=$3mA$] ++(0,4)
	%to[short] ++(-2,0);
	\end{circuitikz}
	\end{center}

And so we can calculate $R_{eq}=R\parallel(R+R+R)=\frac{1}{\frac{1}{R}+\frac{1}{3R}}=\frac{1}{\frac{4}{3R}}=\frac{3}{4}R$


}
\qitem{
	Consider the following circuit. Find the Thevenin and Norton equivalent circuits.

    %feel free to modify this
	\begin{center}
	\begin{circuitikz}

	\draw(0,4)
	to[short] ++(0,-1)
	to[V_=$12V$] ++(0,-2) 
	to[short] node[ground] {} ++(0,-2);
	
	\draw(0,4)
	to[R, l = $10\Omega$] ++(2, 0)
	to[cI, l_= $1.5i$, invert] ++(0, -4)
	to[short] ++(-2,0);
	
	\draw(2,4)
	to[R, l = $6\Omega$, i = $i$] ++(2, 0)
	to[short, *-o] ++(1,0)
	to node[right] {$A$} ++(0,0);
	
	\draw(4, 0)
	to[short, *-o] ++(1,0)
	to node[right] {$B$} ++(0,0);
	
	\draw(2, 0)
	to[short] ++(2, 0)
	to[R, l = $8\Omega$] ++(0,4);
	
	%https://www.allaboutcircuits.com/technical-articles/thevenin-theorem-dependent-source-circuits/

	\end{circuitikz}
	\end{center}
	
% 	\begin{enumerate}[(i)]
% 	    \item First, find $V_{OC}$
% 	    \item Then, find $I_{SC}$
% 	\end{enumerate}
}

\ans{

Let us define some additional currents and voltages so that we can perform nodal analysis:
	\begin{center}
	\begin{circuitikz}

	\draw(0,4)
	to[short] ++(0,-1)
	to[V_=$12V$] ++(0,-2) 
	to[short] node[ground] {} ++(0,-2);
	
	\draw(0,4)
	to[R, l = $10\Omega$, i_ = $i_1$] ++(2, 0)
	to node[above] {$V_1$} ++(0,0)
	to[cI, l_= $1.5i$, invert] ++(0, -4)
	to[short] ++(-2,0);
	
	\draw(2,4)
	to[R, l = $6\Omega$, i_ = $i$] ++(2, 0)
	to node[above] {$V_{oc}$} ++(0,0)
	to[short, *-o] ++(1,0);
	
	\draw(4, 0)
	to node[right] {$-$} ++(0,0)
	to[short, *-o] ++(1,0)
	to node[right] {$-$} ++(0,0);
	
	\draw(2, 0)
	to[short] ++(2, 0)
	to[R, l = $8\Omega$] ++(0,4);
	
	%https://www.allaboutcircuits.com/technical-articles/thevenin-theorem-dependent-source-circuits/

	\end{circuitikz}
	\end{center}


    i) First, we find the Thevenin voltage, which is the open circuit voltage labeled $V_{oc}$ across the $8\Omega$ resistor. \\

    Let us perform nodal analysis at $V_2$;
    \\KCL gives us:
    $$i_1 + 1.5i = i$$
    \\From Ohm's Law, we have
    $$i_1 = \frac{12V-V_1}{10\Omega}$$
    \\$$i = \frac{V_1 - V_{oc}}{6\Omega} = \frac{V_{oc}}{8\Omega}$$.

    \\Note that the current across the $6\Omega$ resistor is the same as the current across the $8\Omega$ resistor because resistors in series share a current.
    \\First, let us plug in our current values into the KCL equation:
    \begin{align}
        \frac{12V-V_1}{10\Omega} + 1.5\frac{V_{oc}}{8\Omega} = \frac{V_{oc}}{8\Omega}    \label{ex}
    \end{align}
    \\Then let's solve for $V_1$ in terms of $V_{oc}$:
    \\$$V_1 - V_{oc} = \frac{6V_{oc}}{8\Omega}$$
    \\$$V_1 = \frac{6V_{oc}}{8\Omega} + V_{oc} = \frac{14V_{oc}}{8\Omega}$$
    \\Then, substitute $V_1$ into equation \eqref{ex}:
    \\$$\frac{12V-\frac{14V_{oc}}{8\Omega}}{10\Omega} + 1.5\frac{V_{oc}}{8\Omega} = \frac{V_{oc}}{8\Omega}$$
    \\Solving for $V_{oc}$, we get $V_{oc} = \frac{96}{9}V$, or approximately $10.67V$.
    
    
    ii) Next, we want to find $I_{sc}$ at the terminals.
We want to create a short at the terminals.
	\begin{center}
	\begin{circuitikz}

	\draw(0,4)
	to[short] ++(0,-1)
	to[V_=$12V$] ++(0,-2) 
	to[short] node[ground] {} ++(0,-2);
	
	\draw(0,4)
	to[R, l = $10\Omega$] ++(2, 0)
	to[cI, l_= $1.5i$, invert] ++(0, -4)
	to[short] ++(-2,0);
	
	\draw(2,4)
	to[R, l = $6\Omega$, i = $i$] ++(2, 0)
	to[short, *-o] ++(1,0)
    to[short, i = $I_{sc}$] ++(0, -4);

	\draw(4, 0)
	to[short, *-o] ++(1,0)
	to node[right] {$B$} ++(0,0);
	
	\draw(2, 0)
	to[short] ++(2, 0)
	to[R, l = $8\Omega$] ++(0,4);
	
	%https://www.allaboutcircuits.com/technical-articles/thevenin-theorem-dependent-source-circuits/

	\end{circuitikz}
	\end{center}


Then, the resistor is no longer on the path of the current. Let us also define some additional currents and voltages so that we can perform nodal analysis:
	\begin{center}
	\begin{circuitikz}

	\draw(0,4)
	to[short] ++(0,-1)
	to[V_=$12V$] ++(0,-2) 
	to[short] node[ground] {} ++(0,-2);
	
	\draw(0,4)
	to[R, l = $10\Omega$, i_ = $i_1$] ++(2, 0)
	to node[above] {$V_1$} ++(0,0)
	to[cI, l_= $1.5i$, invert] ++(0, -4)
	%to node[below] {$V_2$} ++(0,0)
	to[short] ++(-2,0);
	
	\draw(2,4)
	to[R, l = $6\Omega$, i = $i$] ++(2, 0)
	to[short, *-o] ++(1,0)
    to[short, i = $I_{sc}$] ++(0, -4);

	\draw(4, 0)
	to[short, *-o] ++(1,0)
	to node[right] {$B$} ++(0,0);
	
	\draw(2, 0)
	to[short] ++(2, 0);
	
	%https://www.allaboutcircuits.com/technical-articles/thevenin-theorem-dependent-source-circuits/

	\end{circuitikz}
	\end{center}

Now, using KCL at $V_1$, we get the equation:
\\$$i_1 + 1.5i = I_{sc}$$
\\Using Ohm's law, we can plug in the values for $i_1$:
\\$$\frac{12V - V_1}{10} = i_1$$
Also, note that $i = I_{sc}$
\\Solve this equation for $V_1$.
\\$$\frac{12V - V_1}{10} + 1.5\frac{V_1}{6}= \frac{V_1}{6}$$
\\$$V_1 = 72V$$
Then, solve for $I_{sc}$: $I_{sc} = i = \frac{V_1}{6}$
\\$I_{sc} = \frac{72}{6} = 12A$

\\Now, we can solve for $R_{eq}$: $R_{eq} = \frac{V_oc}{I_sc} = \frac{96/9}{12} = 0.889$
}

\qitem{
Now, draw the Thevenin and Norton Equivalent Circuit
}

\ans{
Thevenin Equivalent Circuit:
\begin{center}
	\begin{circuitikz}

	\draw(0,4)
	to[short] ++(0,-1)
	to[V_, l=$V_{oc} \text{=10.67V}$] ++(0,-2)
	to[short] node[ground] {} ++(0,-2)
	to[short] ++(0,1);
	
	
	\draw(0,4)
	to[R = $.889\Omega$, *-o] ++(2, 0)
	to node[right] {$A$} ++(0,0);
	
	
	\draw(0, 0)
	to[short, *-o] ++(2,0)
	to node[right] {$B$} ++(0,0);
	
	\end{circuitikz}
	
	\end{center}
	
\\Norton Equivalent Circuit:
	\begin{center}
	
	\begin{circuitikz}

	\draw(0,4)
	to[short] ++(0,-1)
	to[I, l_=$I_{sc} \text{=}12A$, invert] ++(0,-2)
	to[short] node[ground] {} ++(0,-2)
	to[short] ++(0,1);
	
	
	\draw(0,4)
	to[short, *-o] ++(3, 0)
	to node[right] {$A$} ++(0,0);
	
	\draw(2, 4)
	to[R = $.889\Omega$] ++ (0, -4);
	
	\draw(0, 0)
	to[short, *-o] ++(3,0)
	to node[right] {$B$} ++(0,0);

	
	\end{circuitikz}
	
	\end{center}
}

\qitem{
You can find $R_{eq}$ by taking the ratio of $\frac{V_{OC}}{I_{SC}}$ from the previous part. But in this part, we will find $R_{eq}$ directly by using test voltages and currents. The idea is to zero out all independent sources in the circuit (not dependent sources though). Then, we attach a test voltage $V_{test}$ between A and B and calculate the value of $I_{test}$, which is the current through your test voltage. $R_{eq}$ will be equal to the ratio $\frac{V_{test}}{I_{test}}$. Find the value of $R_{eq}$ by this method, and compare it with the value of $\frac{V_{OC}}{I_{SC}}$
}

\ans{

	\begin{center}
	\begin{circuitikz}

	\draw(0,4)
	to[short] ++(0,-3)
	to[short] node[ground] {} ++(0,-2);
	
	\draw(0,4)
	to[R, l = $10\Omega$, i_ = $i_1$] ++(2, 0)
	to node[above] {$V_1$} ++(0,0)
	to[cI, l_= $1.5i$, invert] ++(0, -4)
	%to node[below] {$V_2$} ++(0,0)
	to[short] ++(-2,0);
	
	\draw(2,4)
	to[R, l = $6\Omega$, i = $i$] ++(2, 0)
	to[short, *-o] ++(1,0)
    to[V^ = $V_{test}$] ++(0, -4);

	\draw(4, 0)
	to[short, *-o] ++(1,0)
	to node[right] {$B$} ++(0,0);
	
	\draw(2, 0)
	to[short] ++(2, 0)
	to[R, l = $8\Omega$, i_=$ $, invert] ++(0,4);

	\end{circuitikz}
	\end{center}

Applying KCL at the $V_1$ node:
$$\frac{0 - V_1}{10} + 1.5i = i$$
$$\frac{V_1 - V_{test}}{6} = i$$

Solving, we get
$$\frac{V_1}{12} - \frac{V_1}{10} - \frac{V_{test}}{12} = 0$$
$$\implies \frac{-V_1}{60} = \frac{V_{test}}{12}$$
$$\implies V_1 = -5V_{test}$$

Now, we can solve for the current between A and B.
$$i = \frac{V_1 - V_{test}}{6} = \frac{-5V_{test} - V_{test}}{6} = -V_{test}$$.
$$i_{BA} = \frac{V_{test} - 0}{8} - i = \frac{V_{test}}{8} + V_{test}= \frac{9}{8}V_{test}$$

So, $R_{eq} = \frac{V_{test}}{i_{BA}} = \frac{8}{9} = 0.889$, which is the same as what we got before.
}



\end{enumerate}
