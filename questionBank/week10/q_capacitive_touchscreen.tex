% Author: Varsha Ramakrishnan
% Email:vio@berkeley.edu

\qns{Capacitive Touchscreen}

\sol{Prereq: Ability to solve capacitor equivalences.}



Consider the following capacitive touchscreen configuration. The first image is a side view, and the second image is a top-down view. In a full touchscreen, each of these bars would be accompanied by many more parallel vertical and horizontal bars. (This is a review from lecture.)

Let the distance between the vertical and horizontal bar be $d$.
\begin{center}
  \begin{circuitikz}
      % draw cube
      \draw (0.2, 0.5) -- (0.2, 0.9) -- (-0.2, 0.9) -- (-0.2, 0.5) -- cycle;
      \node[draw=none] at (0.6, 0.7) {$E_1$};
      % horizontal line
      \draw[color=red] (-2, 0.2) -- (2, 0.2) -- (2, -0.2) -- (-2, -0.2) -- cycle; 
      \draw[color=red] (-2, 0) to [short, -o] (-2.5, 0);
      \node[draw=none,text=red] at (-2.5, 0.5) {$E_2$};
      \draw[<->] (-0.2, 0.2) -- (-0.2, 0.5) node[midway, fill=none, text=blue] {};
      \node[draw=none,text=black] at (-0.4, 0.35) {$d$};
  \end{circuitikz}
\end{center}

And let the cross-sectional area be $A$.
\begin{center}
  \begin{circuitikz}
      % vertical line
      \draw (0.2, -2) -- (0.2, 2) -- (-0.2, 2) -- (-0.2, -2) -- cycle; 
      \draw (0, -2) to [short, -o] (0, -2.5);
      \node[draw=none,text=black] at (0.5, -2.5) {$E_1$};

      % horizontal line
      \draw[color=red] (-2, 0.2) -- (2, 0.2) -- (2, -0.2) -- (-2, -0.2) -- cycle; 
      \draw[color=red] (-2, 0) to [short, -o] (-2.5, 0);
      \node[draw=none,text=red] at (-2.5, 0.5) {$E_2$};

      % center cube
      \draw [color=blue, fill=blue] (-0.2, -0.2) -- (-0.2, 0.2) -- (0.2, 0.2) -- (0.2, -0.2) -- cycle;
      \node[draw=none,text=blue] at (0.5, 0.5) {$A$};
  \end{circuitikz}
\end{center}

\begin{enumerate}
\qitem{
Draw a diagram representing the capacitance between $E_1$ and $E_2$ when there is no touch on the screen.
}

\ans{
The capacitor is formed from the intersection of the two bars shown by the blue square on the above diagram.

\begin{center}
\begin{circuitikz}
\draw(0,0) 
    to[short, -o, l=$E_1$] ++(0, 0);
\draw(0,0)
	to[C=$C_1$] ++(0,2)
	to[short, -o, l=$E_2$] ++(0,0);

\end{circuitikz}
\end{center}
}

\qitem {
Calculate the value of the capacitance between the two bars when the screen is not being touched.
}
\ans{
$$C_1 = \epsilon A/d$$
}

\qitem{
	Now consider what happens when we touch the screen. Let the blue line represent our finger, and assume there is a capacitance between your finger and each of the bars. The diagram looks like this:
	
\begin{center}
  \begin{circuitikz}
      % draw top line
      \draw[ultra thick, color=blue, ] (-2, 2.6) -- (2, 2.6);
      % draw finger label
      \node[draw=none,text=blue] at (-2.75, 2.6) {Finger};
      % draw cube
      \draw (0.2, 1.2) -- (0.2, 1.6) -- (-0.2, 1.6) -- (-0.2, 1.2) -- cycle;
       % draw square label
      \node[draw=none] at (-0.5, 1.4) {$E_1$};
      % horizontal line
      \draw[color=red] (-2, 0.2) -- (2, 0.2) -- (2, -0.2) -- (-2, -0.2) -- cycle; 
      \draw[color=red] (-2, 0) to [short, -o] (-2.5, 0);
      \node[draw=none,text=red] at (-2.5, 0.5) {$E_2$};

      % draw capacitors
      \draw (0, 0.2) to [C, l=$C_{\text{no touch}}$] (0, 1.2);
      \draw (0, 1.6) to [C, l=$C_{F-E_1}$] (0, 2.6);
      \draw (1.2, 0.2) to [C] (1.2, 2.6);
      \node[draw=none] at (2.2, 1.3) {$C_{F-E_2}$};
  \end{circuitikz}
\end{center}
    Redraw the circuit diagram representing the capacitive touchscreen after being touched, so that the nodes representing $E_1$ and $E_2$ are on opposite ends of the diagram.
    
\begin{center}
\begin{circuitikz}

	
\draw(0,0)
    to[short, -o, l=$E_2$] ++(0, -1);

\draw(0,4)
    to[short, -o, l=$E_1$] ++(0, 1);

\end{circuitikz}
\end{center}
    
}

\ans{
\begin{center}
\begin{circuitikz}
\draw(0,0) 
	to[short] ++(3,0)
	to[C=$C_{F-E_2}$] ++(0,2)
	to node[right] {$\gets$ Finger} ++(0,0)
	to[C=$C_{F-E_1}$] ++(0,2)
	to[short] ++(-3,0)
	to[C=$C_{no\ touch}$] ++(0,-4);
	
\draw(0,0)
    to[short, -o, l=$E_2$] ++(0, -1);

\draw(0,4)
    to[short, -o, l=$E_1$] ++(0, 1);

\end{circuitikz}
\end{center}
}

\qitem{Calculate the new capacitance between $E_1$ and $E_2$ ($C_{E_1-E_2}$), in terms of $C_{F-E_2}, C_{F-E_1},$ and  $C_{no\ touch}$. Has it changed from when there was no touch? \\ 
Note: You can use the parallel operator $||$ in your equation, but note that $||$ is a \textit{math} operator, and may not necessarily refer to capacitors in parallel.
}

\ans{
Begin by writing the equation that relates all the capacitors using parallel and series symbols, then expand:
\begin{align*}
C_{E_1-E_2}  &= C_{no\ touch} + C_{F-E_1} || C_{F-E_2} \\
&= C_{no\ touch} + \frac{C_{F-E_1}C_{F-E_2}}{C_{F-E_1} + C_{F-E_2}}
\end{align*}
The new capacitance (after touch) is larger than the previous capacitance (before touch). Since the capacitance value has changed by a measurable amount, we can work backwards and find the distance at which the screen was touched.
}


\end{enumerate}