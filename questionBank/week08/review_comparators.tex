% Lydia Lee, Spring 2019, lydia.lee@berkeley.edu
% \sol{
\qns{Review: Comparators}

Operational amplifiers (op amps for short) are typically drawn like the figure on the left, and their internal workings can be represented by the figure on the right.

\begin{minipage}{0.45\linewidth}
\begin{center}
  \begin{circuitikz}
    \draw (0,0) node[op amp,yscale=-1] (opamp) { }
      (-2.5, 0.5) to [short, i=$i_{+}$, o-] (opamp.+) 
      (-2.5, -0.5) to [short, i=$i_{-}$, o-] (opamp.-) 
      (opamp.out) node[right ] {$U_{out}$} 
      ;
    \node[draw=none,text=black] at (-2.8, 0.5) {$u_{+}$};
    \node[draw=none,text=black] at (-2.8, -0.5) {$u_{-}$};

    \draw (0, 0.5) to [short, -o] (0, 1.0);
    \node[draw=none,text=black] at (0, 1.5) {$V_{DD}$};
    \draw (0, -0.5) to [short, -o] (0, -1.0);
    \node[draw=none,text=black] at (0, -1.5) {$V_{SS}$};
  \end{circuitikz}
\end{center}
\end{minipage}
\begin{minipage}{.45\linewidth}
  \begin{center}
    % \textcolor{red}{This diagram may change slightly from semester to semester. We should stay consistent with the course notes to not confuse students.}
    \begin{circuitikz}
      % Nodes on the left
      \node[draw=none,text=black] at (-1, 0.8) {$u_{+}$};
      \node[draw=none,text=black] at (-1, -0.8) {$u_{-}$};
      \node[draw=none,text=black] at (-.8, 0.3) {$+$};
      \node[draw=none,text=black] at (-.8, -0.3) {$-$};
      \draw (-1.5, 0.5) to [short, -o] (-1, 0.5); 
      \draw (-1.5, -0.5) to [short, -o] (-1, -0.5); 
      
      % ground and variable voltage
      \draw (0.5,-1.5) -- (0.5,-1.8) to [short, -o](0.5, -2);
      \node[draw=none,text=black] at (1, -2) {$V_{SS}$};
      \draw (0.5, -1.5) -- (1, -1) -- (0.5, -0.5) -- (0, -1) -- cycle;
      \node[draw=none,text=black] at (0.5, -0.8) {$+$};
      \node[draw=none,text=black] at (0.5, -1.2) {$-$};
      \node[draw=none,text=black] at (1.8, -1) {$\frac{V_{DD}-V_{SS}}{2}$};
      \draw (0.5, -0.5) -- (0.5, 0) { };
      \draw (0.5, 0) -- (1, 0.5) -- (0.5, 1) -- (0, .5) -- cycle;
      \node[draw=none,text=black] at (0.5, 0.7) {$+$};
      \node[draw=none,text=black] at (0.5, 0.3) {$-$};
      \node[draw=none,text=black] at (2, 0.5) {$A(u_{+} - u_{-})$};
      
      % Upwards
      \draw (0.5, 1) -- (0.5, 1.5) to [short, -o] (3.5, 1.5);
      \draw (3.5,-1.5) to [short, o-] (3.5,-1.8) node[ground]{ };
      \node[draw=none,text=black] at (3.5, 1.8) {$U_{out}$};
      \node[draw=none,text=black] at (3.9, 1) {$+$};
      \node[draw=none,text=black] at (3.9, -1.6) {$-$};
    \end{circuitikz}
  \end{center} 
\end{minipage}

Here, $u_+$ and $u_-$ (both potentials) are input signals, $V_{DD}$ and $V_{SS}$ are what we call the ``supply rails'', and $U_{out}$ (a potential) is the output signal. From the diagram and knowing that $U_{out}$ cannot exceed the supply rail voltages, we have a relationship between the outputs and the inputs:
\begin{equation*}
U_{out} =
\begin{cases} 
V_{SS} &  
\hspace{1.4cm}A(u_+ - u_-) + \frac{V_{DD} +V_{SS} }{2} < V_{SS} \\
A(u_+ - u_-) + \frac{V_{DD} +V_{SS} }{2} & 
\hspace{0.4cm}V_{SS}\leq A(u_+ - u_-) + \frac{V_{DD} +V_{SS} }{2} \leq V_{DD} \\
V_{DD} &  
\hspace{0.4cm}V_{DD}<A(u_+ - u_-) + \frac{V_{DD} +V_{SS} }{2} \\
\end{cases}
\end{equation*}

\begin{center}
\begin{tikzpicture}
  \begin{axis}[
    xmin=-120, xmax=120,
    ymin=-120, ymax=120,
    axis lines=center,
    axis on top=true,
    grid style={dashed},
    xlabel={\emph{$u_{+} - u_{-}$}},
    ylabel={\emph{$U_{out}$}},
    yticklabels={,,},
    xticklabels={,,},
  ]
  \addplot [draw=blue,thick] coordinates {(5, 70) (100, 70)};
  \addplot [draw=blue,thick] coordinates {(-5, -20) (-100, -20)};
  \addplot [draw=blue,thick] coordinates {(-5, -20) (5, 70)};
  \addplot [
      mark=*, blue, solid
    ]
    coordinates {
            (-5, -20)
            (5, 70)};

    \node[anchor=north west] at (axis cs:-5,-20) {$(0,V_{SS})$};
    \node[anchor=south east] at (axis cs:5,70) {$(0,V_{DD})$};
  \end{axis}
\end{tikzpicture}
\end{center}

Typically $A$ is quite large, meaning the the sloped region in the center is quite narrow. This is particularly useful when comparing two voltages (hence, the name \textit{comparator}), one or both of which may be variable.
% }
\empt{\newpage}