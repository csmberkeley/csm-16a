% Lydia Lee, lydia.lee@berkeley.edu
% Modified from Fall 2018 EE16A-in-a-day review for integrators
\qns{Integration By Caps}
\begin{enumerate}

\qitem\label{ques:cv} How much charge $Q_+$ is accumulated on the top plate of the capacitor?
\begin{center}
	\begin{circuitikz}
		\draw
		(0,0) to[C,l=$C$, v=$V_{C}$] ++(0,-2);
	\end{circuitikz}
\end{center}

\ans{
	\begin{align*}
		Q_+ &= CV_C
	\end{align*}
}

\qitem\label{ques:negcv} For the circuit in part \ref{ques:cv}, how much charge $Q_-$ is accumulated on the bottom plate of the capacitor?
\empt{
	\vspace{1cm}
}
\sol{
	If the semester's offering covers charge sharing and/or switched capacitors, the distinction between parts \ref{ques:cv} and \ref{ques:negcv} is very important!
}

\ans{
	\begin{align*}
		Q_- &= -CV_C
	\end{align*}
}

\qitem\label{ques:constant_current}For the following subparts, use the following circuit
	\begin{center}
		\begin{circuitikz}
		    \draw(0,0) to[I,l=$I_S(t)$,invert] ++(0,2)
		        to[short] ++(2,0)
		        to[C=$C$,v^=$V_{C}$] ++(0,-2)
		        to[short] (0,0);
	    \end{circuitikz}
	\end{center}
	Find $V_C(t)$ given 
	\begin{itemize}
		\item $V_C(t=0) = V_0$ 
		\item $I_S(t) = I_0$ (constant with respect to time)
		\item $C$ is constant with respect to time
	\end{itemize}
\empt{
	\vspace{3cm}
}
\ans{
	First, we recall the expression relating current and voltage for a capacitor:
	\begin{align*}
		Q_+ &= CV_C\\
		\frac{dQ_+}{dt} &= I_C\\
			&= C\frac{dV_C}{dt} + V_C\frac{dC}{dt}\\
			&= C\frac{dV_C}{dt}\\
	\end{align*}
	However, note that the current source points in the \textit{opposite} direction of $V_C$ with regard to passive sign convention!
	\begin{align*}
		I_C &= -I_S(t)\\
		C\frac{dV_C}{dt} &= -I_0\\
		\int_{V_C(t=0)}^{V_C(t)}dV_C &= \int_{\tau=0}^{\tau=t}\frac{-I_0}{C}d\tau\\
		V_C(t) - V_C(t=0) &= -\frac{I_0}{C}(t-0)\\
		V_C(t) &= V_C(t=0) - \frac{I_0}{C}t\\
			&= V_0 - \frac{I_0}{C}t
	\end{align*}
}

\qitem\label{ques:double_C}What happens to the rate of change of $V_C$ if the capacitance doubles?
\empt{
	\vspace{1cm}
}
\ans{
	Looking at our expression
	$$V_C(t) = V_0 - \frac{I_0}{C}t$$
	The slope of $V_C(t)$ halves; that is, $V_C$ changes at half its original rate.

}
\empt{
	\newpage
}
\qitem\label{ques:general_integrator}Find $V_C(t)$ given the following. Note that $I_S(t)$ is not necessarily constant! \textit{Hint: You can use integrals}
\begin{itemize}
	\item $V_C(t=0) = V_0$
	\item $C$ is constant with respect to time
\end{itemize}
\empt{
	\vspace{3cm}
}
\ans{
	The process here is fairly similar to part \ref{ques:constant_current}, only now we can't assume $I_S(t)$ is constant.
	\begin{align*}
		I_S(t) &= -I_C\\
			&= -C\frac{dV_C}{dt}\\
		\int_{V_C(t=0)}^{V_C(t)}dV_C &= -\int_{\tau=0}^{\tau=t}\frac{I_S(\tau)}{C}d\tau\\
		V_C(t) - V_C(0) &= -\frac{1}{C}\int_{\tau=0}^{\tau=t}I_S(\tau)d\tau\\
		V_C(t) &= V_0 - \frac{1}{C}\int_{\tau=0}^{\tau=t}I_S(\tau)d\tau
	\end{align*}
}

\qitem\label{ques:constant_current_C}\textbf{(PRACTICE)} Using your expression from part \ref{ques:constant_current}, find the value of the capacitor given
	\begin{itemize}
		\item $V_0 = 1\si{\volt}$
		\item $I_0 = 2\si{\milli\ampere}$ (constant)
		\item $V_C(1\si{\milli\second}) = 0\si{\volt}$
	\end{itemize}
\empt{
	\vspace{2cm}
}
\sol{

	\begin{center}
		\begin{tabular}{|c|c|c|}
			\hline
			Prefix & Symbol & Order of Magnitude\\\hline
			tera & $\si{\tera}$ & $10^{12}$\\\hline
			giga & $\si{\giga}$ & $10^{9}$\\\hline
			mega & $\si{\mega}$ & $10^{6}$\\\hline
			kilo & $\si{\kilo}$ & $10^{3}$\\\hline
			milli & $\si{\milli}$ & $10^{-3}$\\\hline
			micro & $\si{\micro}$ & $10^{-6}$\\\hline
			nano & $\si{\nano}$ & $10^{-9}$\\\hline
			pico & $\si{\pico}$ & $10^{-12}$\\\hline
		\end{tabular}
	\end{center}
	\textbf{Common Mistakes:}
	\begin{itemize}
		\item Writing $\si{\mega}\, (\text{mega, }10^6)$ in lieu of $\si{\milli}\, (\text{milli, }10^{-3})$
	\end{itemize}

}
\ans{
	Using the answer from above:
	$$V_C(t) = V_0 - \frac{I_0}{C}t$$
	we can rearrange the equation to isolate $C$, then plug in values to solve:
	\begin{align*}
		C &= \frac{I_0}{V_0 - V_C(t)}t\\
			&= \frac{2\si{\milli\ampere}}{1\si{\volt}-0\si{\volt}}\cdot 1\si{\milli\second}\\
			&= (2 \cdot 1)(10^{-3}\cdot 10^{-3})\si{\farad}\\
			&= 2\si{\micro\farad}
	\end{align*}
}



\qitem\label{ques:tcross}\textbf{(CHALLENGE PRACTICE)} Using your expression from part \ref{ques:general_integrator} and the table below, determine for each of the following the values of $t$ where $V_C(t) = 0\si{\volt}$.
\begin{center}
	\begin{tabular}{|c|c|}
		\hline
		Parameter & Value\\\hline
		$V_0$ & $0\si{\volt}$\\\hline
		$C$ & $1\si{\nano\farad}$\\\hline
	\end{tabular}
\end{center}
\begin{enumerate}
	\item\label{ques:tcross_constI} $I_S(t) = 1\,\si{\ampere}$
	\item\label{ques:tcross_steps} $I_S(t) &= \begin{cases}
						1\si{\ampere} & t \leq 1\si{\second}\\
						-1\si{\ampere} & t > 1\si{\second}
					\end{cases}$
	\item\label{ques:tcross_sinusoid} $I_S(t) = \begin{cases}
						\sin(t)\,\si{\ampere} & t \in [0,2\pi]\\
						0\si{\ampere} & \text{otherwise}
					\end{cases}$
\end{enumerate}
\empt{
	\newpage
}
\ans{
	
	For subparts \ref{ques:tcross_constI} and \ref{ques:tcross_steps}, it's possible to plot $V_C(t)$ and compute where the voltage is zero. However, this becomes a lot more difficult for \ref{ques:tcross_sinusoid}!

	Consider our expression from \ref{ques:general_integrator}
	$$V_C(t) = V_0 - \frac{1}{C}\int_{\tau=0}^{\tau=t}I_S(\tau)d\tau$$
	We're interested in $V_C(t) = 0$, so plugging in our values
	\begin{align*}
		V_C(t) &= -\frac{1}{C}\int_{\tau=0}^{\tau=t}I_S(\tau)d\tau = 0\\
		0 &= \int_{\tau=0}^{\tau=t}I_S(\tau)d\tau
	\end{align*}
	With a bit of manipulation, we can see that all we \textit{really} care about is when the integral of $I_S(t)$ with respect to time is zero.

	Abusing units (sorry about that), we can solve this numerically or use the graphical interpretation of the integral as the area under a curve.
	\begin{enumerate}
		\item $t = 0\si{\second}$. Numerically,
		\begin{align*}
			0 &= \int_{\tau=0}^{\tau=t}I_S(\tau)d\tau\\
				&= \int_{\tau=0}^{\tau=t}1(\tau)d\tau\\
				&= t
		\end{align*}
		\item $t = \{0, 2\}\si{\second}$. Numerically,
		\begin{align*}
			0 &= \int_{\tau=0}^{\tau=1\si{\second}}1d\tau - \int_{\tau=1\si{\second}}^{\tau=t}1d\tau\\
				&= (1-0) - (t-1)\\
				&= -t + 2
		\end{align*}
		Graphically, 
		\begin{center}
			\begin{tikzpicture}
\begin{axis}[
    axis lines = middle,
    xlabel = {$t$ $(\si{\micro\second})$},
    ylabel = {$I_S(t)$},
    xtick={0,1,2,3},
    ytick={0,-1,1},
    ymin=-1.5,
    ymax=1.5
]
\addplot [
    color=black
    ]
    coordinates {
        (0,1)
        (1,1)
        (1,-1)
        (3,-1)
    };

    \node[anchor=west] at (axis cs:0,1) {$I_1$};
    \node[anchor=west] at (axis cs:0,-1) {$-I_1$};
 
\end{axis}
\end{tikzpicture}
		\end{center}
		\item $t = \{\leq 0, \geq 2\pi\} $\si{\second}. Numerically,
		\begin{align*}
			0 &= \int_{\tau=0}^{\tau=t\leq 2\pi}\sin(t)dt\\
				&= -\cos(t) + 1 \text{ over the interval }[0,2\pi]\\
		\end{align*}
		Graphically, the integrated area under the curve is zero only when $t\leq 0$ and $t \geq 2\pi$
		\begin{center}
			\begin{tikzpicture}
\begin{axis}[
    axis lines = middle,
    xlabel = {$t$ $(\si{\second})$},
    ylabel = {$I_S(t)$},
    xtick={0,1.57,3.14,4.71,6.28, 7.85, 9.42, 11.00, 12.57},
    xticklabels={$0$, $\frac{\pi}{2}$,$\pi\,$,$\,\,\,\frac{3}{2}\pi$,$\,\,\,2\pi$, $\frac{5}{2}\pi$, $3\pi$, $\frac{7}{2}\pi$, $4\pi$},
    ytick={0,-1,1},
    ymin=-1.5,
    ymax=1.5,
    xmin = 0,
    xmax=4*pi
]
\addplot [
	color=black,
	domain=0:2*pi,
	samples=200
	]
	{ sin(deg(x))  
	};
\end{axis}
\end{tikzpicture}
		\end{center}
	\end{enumerate}

}
\end{enumerate}