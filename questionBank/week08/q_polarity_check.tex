% Lydia Lee, modified from EE16A-In-A-Day Fall 2018
\qns{Polarity Check}\ \\
Determine if the following systems are in negative feedback.
\begin{enumerate}
\qitem\ \\
	\begin{center}
		\begin{circuitikz}
	\draw
	(0,0) node[op amp] (AMP) {}
	(AMP.-) to[short] ++(0,1) coordinate (topLeft)
		to[R,l=$R_f$] (topLeft -| AMP.out)
		to[short] (AMP.out)
		to[short,-o] ++(1,0)
		to[open,o-o,v^=$v_\text{out}$] ++(0,-2)
		node[ground] () {}
	(AMP.-) to[R,l_=$R_s$] ++(-2,0)
		to[sV,v_=$v_\text{in}$] ++(0,-2)
		node[ground] () {}
	(AMP.+) to[short] ++(0,-1)
		node[ground] () {};
\end{circuitikz}
	\end{center}
\ans{
	This is in negative feedback, and in fact is a fairly well-known topology called the inverting amplifier.
	\begin{itemize}
		\item For our initial stimulus, we'll kick $v_\text{out}$ up
		\item How does $u_-$ respond? Well, we can't assume $u_+ = u_-$ (because that requires the system to be in negative feedback), so we'll look at the resistors. Note that this is a voltage divider with $u_-$ in the and, $v_\text{out}$ at the top, so when $v_\text{out}$ moves upward, $u_-$ travels upward too
		\item If $u_-$ goes up, that means $v_\text{out}$ goes down because of the op amp---the opposite direction as the iniital stimulus!
	\end{itemize}
}

\qitem\ \\
	\begin{center}
		\begin{circuitikz}
	\draw
	(0,0) node[op amp] (AMP) {}
	(0,2) node[buffer,xscale=-1] (FB) {}
	(AMP.out) to[short] (AMP.out |- FB.in)
		to[short] (FB.in)
	(FB.out) to[short] (FB.out -| AMP.-)
		to[short] (AMP.-)
		to[R,l=$R$] ++(-2,0)
		to[sV,v_=$v_\text{in}$] ++(0,-2)
		node[ground] () {}
	(AMP.+) to[short] ++(-.5,0)
		node[ground] () {}
	(AMP.out) to[short] ++(1,0)
		to[open,o-o,v^=$v_\text{out}$] ++(0,-1)
		node[ground] () {}
	(FB) node[] () {-5};
\end{circuitikz}
	\end{center}
\ans{
	This is \textit{not} in negative feedback! Going through the process of checking:
	\begin{itemize}
		\item For our initial stimulus, we'll kick $v_\text{out}$ up
		\item With the $-5$, that means $u_-$ goes down
		\item If $u_-$ goes down, that means $v_\text{out}$ goes up---the same direction as the initial stimulus!
	\end{itemize}
}


\qitem\ \\
	\begin{center}
		\begin{circuitikz}
	\draw
	(0,0) node[op amp,yscale=-1] (AMP) {}
	(0,2) node[buffer,xscale=-1] (FB) {}
	(AMP.out) to[short] (AMP.out |- FB.in)
		to[short] (FB.in)
	(FB.out) to[short] (FB.out -| AMP.+)
		to[short] (AMP.+)
		to[R,l=$R$] ++(-2,0)
		to[sV,v_=$v_\text{in}$] ++(0,-2)
		node[ground] () {}
	(AMP.-) to[short] ++(-.5,0)
		node[ground] () {}
	(AMP.out) to[short] ++(1,0)
		to[open,o-o,v^=$v_\text{out}$] ++(0,-1)
		node[ground] () {}
	(FB) node[] () {-5};
\end{circuitikz}
	\end{center}

\ans{
	This is indeed in negative feedback! Let's go through the process of checking:
	\begin{itemize}
		\item For our initial stimulus, we'll kick $v_\text{out}$ up
		\item With the $-5$, that means $u_+$ goes down
		\item If $u_+$ goes down, that means $v_\text{out}$ goes down---the opposite direction of the initial stimulus!
	\end{itemize}
}
\end{enumerate}