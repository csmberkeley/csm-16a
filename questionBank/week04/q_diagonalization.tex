% Author: Yannan Tuo, Varsha Ramakrishnan
% Email: ytuo@berkeley.edu, vio@berkeley.edu

\qns{Diagonalization and Other Things Related (ish)}

\begin{enumerate}

\qitem{When is an $n$ x $n$ matrix diagonalizable, or able to be represented in the form $\mathbf{P}\mathbf{D}^n\mathbf{P}^{-1}$, where $\mathbf{D}$ is a diagonal matrix?}

\ans{A square matrix of size n is diagonalizable when it has $n$ linearly independent eigenvectors, or when the matrix formed by the eigenvectors is full rank. (Note that the eigenvalues do not need to be unique)}


\qitem{Given eigenvalues $\lambda = 1, 2$, diagonalize this matrix 
 $$ \mathbf{A = \begin{bmatrix}
  2 & 0 & 0 \\
  1 & 2 & 1 \\
  -1 & 0 & 1
 \end{bmatrix}}$$
 \\
}

\meta{
[Notice]: A mini lecture may be required before going over the diagonalization problem because students may not have seen this in lecture yet.
For students interested, eigenvalues can be calculated by solving for when $det(A - \lambda I) = 0$ through cofactoring (not yet taught)
$$det\mathbf{\begin{bmatrix}
  2-\lambda & 0 & 0 \\
  1 & 2-\lambda & 1 \\
  -1 & 0 & 1-\lambda
 \end{bmatrix}}  =  (2 - \lambda)^{2}(1-\lambda)  =  0$$
 $$ \lambda = 1, \lambda = 2 \text{ (2 values)}$$
 }

\ans{

 
Step 1: Find linearly independent eigenvectors of A by solving for the nullspace of $(A-\lambda I)$ for each value of $\lambda$
 $$ \lambda=1 : \mathbf{\begin{bmatrix}
  2-\lambda & 0 & 0 \\
  1 & 2-\lambda & 1 \\
  -1 & 0 & 1-\lambda
 \end{bmatrix}} = 
 \mathbf{\begin{bmatrix}
  1 & 0 & 0 \\
  1 & 1 & 1 \\
  -1 & 0 & 0
 \end{bmatrix}}
 $$
 $$ \hat{v} = \mathbf{\begin{bmatrix}
  0\\
  -1\\
  1
 \end{bmatrix}}
 $$
 
 $$ \lambda=2 : \mathbf{\begin{bmatrix}
  2-\lambda & 0 & 0 \\
  1 & 2-\lambda & 1 \\
  -1 & 0 & 1-\lambda
 \end{bmatrix}} = 
 \mathbf{\begin{bmatrix}
  0 & 0 & 0 \\
  1 & 0 & 1 \\
  -1 & 0 & -1
 \end{bmatrix}}
 $$
 $$ \hat{v} = \mathbf{\begin{bmatrix}
  0\\
  1\\
  0
 \end{bmatrix}}, \mathbf{\begin{bmatrix}
  -1\\
  0\\
  1
 \end{bmatrix}}
 $$
 
 Step 2: Arrange the eigenvectors and eigenvalues into the $\mathbf{P}$ and $\mathbf{D}$ matrices. Note: Make sure to match the row and column of the eigenvalues in the $\mathbf{D}$ matrix with the column of the eigenvectors in the $\mathbf{P}$ matrix.

$$ \mathbf{P = \begin{bmatrix}
  0 & 0 & -1 \\
  -1 & 1 & 0 \\
  1 & 0 & 1
 \end{bmatrix}}
$$
$$
 \mathbf{D = \begin{bmatrix}
  1 & 0 & 0 \\
  0 & 2 & 0 \\
  0 & 0 & 2
 \end{bmatrix}}
 $$
}


\qitem{Consider a pump system with transition matrix $\mathbf{A}$, diagonalized as $\mathbf{P}\mathbf{D}\mathbf{P}^{-1}$. Find the system state $\vec{s}[n]$ given state $\vec{s}[0]$
}

\ans{
\\Use the formula $\vec{s}[n] = \mathbf{A}^n\vec{s}[0]$. We want to calculate $(\mathbf{P}\mathbf{D}\mathbf{P}^{-1})^n\vec{s}[0]$, or 
        $\mathbf{P}\mathbf{D}\mathbf{P}^{-1}\mathbf{P}\mathbf{D}\mathbf{P}^{-1}...\mathbf{P}\mathbf{D}\mathbf{P}^{-1}\vec{s}[0]$. We see that the inner $\mathbf{P}^{-1}$ and $\mathbf{P}$'s cancel out to the identity matrix, leaving us with $\mathbf{P}\mathbf{D}^n\mathbf{P}^{-1}\vec{s}[0]$. 
        \\Consider a sizable matrix $\mathbf{A}$, eg $10 \times 10$, and a large exponent $n$, eg 7. It would generally be computationally simpler to diagonalize the matrix and compute $\mathbf{P}\mathbf{D}^n\mathbf{P}^{-1}\vec{s}[0]$ than to compute $\mathbf{A}^n\vec{s}[0]$ because $\mathbf{D}^n$ involves just raising each number in the diagonal to the $n$th power.
}


\qitem{Is there a relationship between invertibility and diagonizability?
    \begin{enumerate} [i]
        \item  First, let us consider: does invertibility imply diagonizability? Give a brief explanation or counterexample. (Hint: think about how linear independence plays a role in whether or not a matrix is invertible or diagonizable).
    
    \item Does diagonizability imply invertibility? (Hint: think about the invertibility of each individual matrix that constitutes the diagonalized matrix.) %(Note that can't really be used: the determinant of a matrix A is equal to the product of its eigenvalues. If not proven in class, you must prove this yourself.)
    
    \end{enumerate}
}

\ans{
\begin{enumerate} [i]
        \item 
        \\No, we cannot assume that a matrix that is invertible is diagonalizable.
         \\When a matrix is invertible, it must have linearly independent columns.
         \\For example, take the matrix A and its inverse:
         $$\mathbf{A} =
            \begin{bmatrix}
            2 & 3 \\
            0 & 2 
            \end{bmatrix}, \ \
            \mathbf{A}^{-1} =
            \begin{bmatrix}
            \frac{1}{2} & -\frac{3}{4} \\
            0 & \frac{1}{2}
            \end{bmatrix}$$
        However, when we solve for the eigenvectors, we only get one: $\begin{bmatrix}
            1 \\
            0
            \end{bmatrix}$.
        \\We need two linearly independent vectors to form a diagonalizable matrix; hence, we have found a matrix that is invertible but not diagonalizable. 
        \\Note: remember that one eigenvalue can map to multiple eigenvectors that are linearly independent, so do not confuse the number of eigenvalues with the number of eigenvectors.
        \item
        \\Consider a diagonalized representation of a matrix, with a diagonal matrix with a 0 entry. This means that one of the eigenvalues of the matrix is a 0. Multiplying $\mathbf{P}\mathbf{D}\mathbf{P}^{-1}$, we would get a column of 0s in the resulting matrix, or a matrix that lacks a pivot and is hence not invertible. For a more trivial solution, consider the $\mathbf{0}$ matrix; not invertible, but diagonalizable.
        %\\Consider a diagonalized representation of a matrix, with a diagonal matrix with a 0 entry. This means that one of the eigenvalues of the matrix is a 0. Then, the product of the eigenvalues is 0, and the determinant of the original matrix is 0. We know that matrices with determinant 0 have no inverse. If this makes your brain hurt, then consider the 0 matrix: not invertible, but diagonalizable.
        \\
        \\Hence, their is no relationship between the properties of invertibility and diagonizability. 
        \end{enumerate}
}


    
    
\end{enumerate}
