% Lydia Lee, lydia.lee@berkeley.edu
\qns{Loaded Question}

\begin{enumerate}

\qitem\label{rin}{
	You're given the following block which we'll call Block 1:
	\begin{center}
		\begin{circuitikz}[scale=0.75, transform shape]
			\draw
			(0,0) node[ground] () {}
				to[sV,l=$v_S$] ++(0,3)
				to[short] ++(1,0)
				to[R=$R_\text{top}$] ++(0,-1.5) coordinate (output)
				to[R=$R_\text{bottom}$] ++(0,-1.5)
				node[ground] () {}

			(output) to[short,*-] ++(2,0)
				node[anchor=south] () {$X$}
				to[open,o-o,v^=$v_X \text{=} v_S\left(\frac{R_\text{bottom}}{R_\text{bottom} + R_\text{top}}\right)$] ++(0,-1.5)
				node[ground] () {};
		\end{circuitikz}
	\end{center}
	For each of the circuits below, connect node $X$ to the output of Block 1. Does the value of $v_X$ change from what it would be otherwise? In other words, does the second block load the voltage divider?

	\begin{enumerate}
		\item \ \\
		\begin{circuitikz}[scale=0.75, transform shape]
			\draw
			(0,0) node[anchor=east] () {$X$}
				to[short,o-] ++(1.5,0)
				to[R=$R_L$] ++(0,-1.5)
				node[ground] () {};
		\end{circuitikz}
		\item \ \\
			\begin{circuitikz}[scale=0.8, transform shape]
	\draw
	(0,0) node[op amp,yscale=-1] (AMP) {}
	(AMP.-) to[short] ++(0,-1) coordinate (bottomLeft)
		to[short] (bottomLeft -| AMP.out)
		to[short] (AMP.out)
		to[short] ++(1,0)
		to[open,o-o,v^=$v_\text{out}$] ++(0,-2)
		node[ground] () {}
	(AMP.+) to[short,-o] ++(-2,0)
		node[anchor=east] () {$X$}
	;
\end{circuitikz}
		\item \ \\
			\begin{circuitikz}[scale=0.8, transform shape]
	\draw
	(0,0) node[op amp,yscale=-1] (AMP) {}
	(AMP.+) to[short,-o] ++(-2,0)
		node[anchor=east] () {$X$}
	(AMP.-) to[short] ++(0,-1.5) coordinate (bottomLeft)
		to[short] (bottomLeft -| AMP.out)
	(AMP.out) to[R,l_=$R_\text{top}$] (bottomLeft -| AMP.out)
		to[R,l_=$R_\text{bottom}$] ++(0,-1.5)
		to[V,l_=$V_\text{REF}$] ++(0,-1.5)
		node[ground] () {}
	(AMP.out) to[short] ++(1,0)
		to[open,o-o,v^=$v_\text{out}$] ++(0,-2)
		node[ground] () {}
	;
\end{circuitikz}
		\item \ \\
			\begin{circuitikz}[scale=0.8, transform shape]
	\draw
	(0,0) node[op amp] (AMP) {}
	(AMP.-) to[short] ++(0,1) coordinate (topLeft)
		to[R,l=$R_f$] (topLeft -| AMP.out)
		to[short] (AMP.out)
		to[short,-o] ++(1,0)
		to[open,o-o,v^=$v_\text{out}$] ++(0,-2)
		node[ground] () {}
	(AMP.-) to[R,l_=$R_s$,-o] ++(-2,0)
		node[anchor=east] () {$X$}
	(AMP.+) to[V,l_=$V_\text{REF}$] ++(0,-2)
		node[ground] () {};
\end{circuitikz}
	\end{enumerate}
	}

\sol{
	What this is asking for can be written in a few equivalent ways:
	\begin{itemize}
		\item Is the Th\'evenin resistance ($R_{\text{in}}$) from the input node of the circuit finite?
		\item Will the circuit in question load the voltage divider above?
		\item Does the circuit draw current from its input node?
	\end{itemize}
	As with all things in circuits, there are a number of different and entirely valid ways to solve this. You can also combine these in various ways to come to the same conclusion.

	\textbf{Method 1: Th\'evenin Equivalent of the Voltage Divider}

		Let's consider the Th\'evenin equivalent circuit of the voltage divider:
		\begin{center}
			\begin{circuitikz}
				\draw
				(0,0) node[anchor=west] () {$v_X$}
					to[R,l_=$R_\text{Th}$,o-] ++(-2,0)
					to[sV,v_=$v_\text{Th}$] ++(0,-2)
					node[ground] () {};
			\end{circuitikz}
		\end{center}
		\begin{align*}
			R_\text{Th} &= R_\text{bottom} || R_\text{top}\\
			v_\text{Th} &= v_S\left(\frac{R_\text{bottom}}{R_\text{bottom} + R_\text{top}}\right)
		\end{align*}
		For each of the circuits, we now replace $v_\text{in}$ with our Th\'evenin equivalent circuit above and see if any current flows from $v_\text{Th}$.
		
	\textbf{Method 2: Th\'evenin Resistance Looking Into the Loading Circuit}

		This particular method is a bit more advanced, but nonetheless good practice. We're interested in the resistance looking \textit{into} the input node of the amplifier topology. That is, if you attach a test voltage at the input of the second circuit, what Th\'evenin resistance do we see? If $R_\text{Th} = \infty$, we know that it doesn't load the voltage divider and so doesn't affect $v_X$. 

		For clarity, relabel your $v_\text{in}$ as $v_\text{test}$ in the diagrams
		\begin{enumerate}
			\item We immediately know that $R_\text{Th} = R_L$, which can be a finite value, so unless $R_L = \infty$, this does affect $v_X$/
			\item $i_\text{test} = 0\si{\ampere}$, so $R_\text{Th} = \frac{v_\text{test}}{0} = \infty\si{\ohm}$! In other words, this looks like an open circuit looking in from its input and it doesn't affect $v_X$.
			\item This problem gets a bit more complicated, but we can still solve it! Applying a test voltage \textbf{with all independent sources turned off} and assuming an ideal op amp for simplicity,
			\begin{align*}
				i_\text{test} &= \frac{v_\text{test}}{R_s}\\
				\frac{v_\text{test}}{i_\text{test}} &= R_s\\
				R_\text{Th} &= R_s
			\end{align*}
			And we see that we have a finite input equivalent resistance! That means that this \textit{will} load the circuit!
			\item Once again, $i_\text{test} = 0\si{\ampere} \Longrightarrow R_\text{Th} = \infty\si{\ohm}$ and it doesn't affect $v_X$/

		\end{enumerate}

	\textbf{Method 3: Brute Force}

		Developing an understanding of how to quickly answer these types of problems by inspection will be extremely useful down the line.
}

\ans{
	\begin{enumerate}
		\item Long story short, this affects $v_X$
		\item This is a voltage buffer! It does not affect the value of $v_X$ of the voltage divider.
		\item This is a noninverting amplifier. Note that no current is drawn from the voltage source because $i_+ = i_- = 0\si{\ampere}$; that said, it does not load the voltage follower and so does not affect the value of $v_X$ from its original value.
		\item This is an inverting amplifier. In this case, current flows across $R_s$, meaning that the amplifier \textit{does} in fact load the voltage divider and affects the value of $v_X$!
	\end{enumerate}
}

\qitem\label{rout}{
	For the following blocks, determine if attaching $R_L$ between the output and ground affects $v_\text{out}$
	\begin{enumerate}
		\item \ \\
			\begin{circuitikz}[scale=0.8, transform shape]
	\draw
	(0,0) node[op amp,yscale=-1] (AMP) {}
	(AMP.-) to[short] ++(0,-1) coordinate (bottomLeft)
		to[short] (bottomLeft -| AMP.out)
		to[short] ++(1.5,0) coordinate (bottomRight)
		to[short] (bottomRight |- AMP.out)
		to[R=$r_o$] (AMP.out)
	(bottomRight |- AMP.out) to[short] ++(1,0)
		to[open,o-o,v^=$v_\text{out}$] ++(0,-2)
		node[ground] () {}
	(AMP.+) to[short] ++(-2,0)
		to[sV,v_=$v_\text{in}$] ++(0,-2)
		node[ground] () {}
	;
\end{circuitikz}
		\item\label{rout:inverting_amp_ro} \ \\
			\begin{circuitikz}[scale=0.8, transform shape]
	\draw
	(0,0) node[op amp] (AMP) {}
	(AMP.-) to[short] ++(0,1) coordinate (topLeft)
		to[R,l=$R_f$] (topLeft -| AMP.out)
		to[short] ++(1.5,0) coordinate (topRight)
		to[short] (topRight |- AMP.out)
		to[R=$r_o$] (AMP.out)
	(topRight |- AMP.out) to[short,-o] ++(1,0)
		to[open,o-o,v^=$v_\text{out}$] ++(0,-2)
		node[ground] () {}
	(AMP.-) to[R,l_=$R_s$] ++(-2,0)
		to[sV,v_=$v_\text{in}$] ++(0,-2)
		node[ground] () {}
	(AMP.+) to[V,l_=$V_\text{REF}$] ++(0,-2)
		node[ground] () {};
\end{circuitikz}
		\item \ \\
			\begin{circuitikz}[scale=0.8, transform shape]
	\draw
	(0,0) node[op amp,yscale=-1] (AMP) {}
	(AMP.-) to[short] ++(0,-1) coordinate (bottomLeft)
		to[short] (bottomLeft -| AMP.out)
		to[short] (AMP.out)
		to[R=$r_\ell$] ++(2,0)
		to[open,o-o,v^=$v_\text{out}$] ++(0,-2)
		node[ground] () {}
	(AMP.+) to[short] ++(-2,0)
		to[sV,v_=$v_\text{in}$] ++(0,-2)
		node[ground] () {}
	;
\end{circuitikz}
		\item \ \\
			\begin{circuitikz}[scale=0.8, transform shape]
	\draw
	(0,0) node[op amp] (AMP) {}
	(AMP.-) to[short] ++(0,1) coordinate (topLeft)
		to[R,l=$R_f$] (topLeft -| AMP.out)
		to[short] (AMP.out)
		to[short,-o] ++(1,0)
		to[open,o-o,v^=$v_\text{out}$] ++(0,-2)
		node[ground] () {}
	(AMP.-) to[R,l_=$R_s$] ++(-2,0)
		to[sV,v_=$v_\text{in}$] ++(0,-2)
		node[ground] () {}
	(AMP.+) to[V,l_=$V_\text{REF}$] ++(0,-2)
		node[ground] () {}
	(AMP.out) to[R=$r_\ell$] ++(0,-2)
		node[ground] () {};
\end{circuitikz}
	\end{enumerate}
}

\sol{
	After enough practice, students should be able to do these types of problems by inspection. Poll your class if they think yes/no. If students ask, you should be ready to work out the full solution. The analysis is below for reference if you don't want to do it on the fly.
	\begin{enumerate}
		\item \ \\
			\begin{align*}
				v_\text{out} &= v_-\\
					&= v_+\\
					&= v_\text{in}
			\end{align*}
		\item \ \\
			\begin{align*}
				\frac{v_\text{in}-V_\text{REF}}{R_s} &= \frac{V_\text{REF}-v_\text{out}}{R_f}
			\end{align*}
			At this point, we can see that we have all the information we need to find $v_\text{out}$, and $r_o$ isn't in it. Ergo, $r_o$ does not have an affect on $v_\text{out}$
		\item Let's call the voltage at the output of the op amp $v_x$.
			\begin{align*}
				v_x &= v_-\\
					&= v_+\\
					&= v_\text{in}\\
				v_\text{out} &= v_x\left(\frac{R_L}{R_L + r_\ell}\right)\\
					&= v_\text{in}\left(\frac{R_L}{R_L + r_\ell}\right)
			\end{align*}
			There is a dependence on $r_\ell$!
		\item This problem is quite similar to part \ref{rout:inverting_amp_ro}, and in fact the analysis is identical! To get some intuition behind this, we can think back to the model of the ideal op amp---that is, that it has an ideal (dependent) voltage source at the input. We know that the variable that said voltage source depends on does not depend on $r_\ell$, and as such its output (which is directly connected to $v_\text{out}$) will not depend on $r_\ell$.
	\end{enumerate}}

\ans{
	\begin{enumerate}
		\item The configuration shown in this part is a voltage buffer. Attaching $R_L$ between output and ground won't affect $v_\text{out}$ because the output voltage remains the same across nodes.
		\item As shown, we essentially have an inverting amplifier for this configuration. However, $r_o$ is within the output node of the amplifier, attaching $R_L$ will have the same effect as in the previous part. The output voltage still remains the same.
		\item Since we connect $r_\ell$ outside the output of the amplifier, once we attach $R_L$ between the end node and the ground, we have essentially built a voltage divider with the 2 resistors, which does have a dependence on $r_\ell$.
		\item For this circuit, when we attach $R_L$ between the output and the ground, we can see that $r_\ell$ and $R_L$ are both in parallel, and the positive end of $v_\text{out}$ can be seen directly connected to the output node of the amplifier, which isn't affected by any resistor(s) along the way. Hence, the output voltage won't change.
	\end{enumerate}}
\end{enumerate}
\empt{\newpage}