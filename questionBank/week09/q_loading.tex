% Lydia Lee, lydia.lee@berkeley.edu
\qns{Loaded Question}

\begin{enumerate}

\qitem\label{rin}{
	Assume you're given the following block:
	\begin{center}
		\begin{circuitikz}[scale=0.75, transform shape]
			\draw
			(0,0) node[ground] () {}
				to[sV,l=$v_S$] ++(0,3)
				to[short] ++(1,0)
				to[R=$R_\text{top}$] ++(0,-1.5) coordinate (output)
				to[R=$R_\text{bottom}$] ++(0,-1.5)
				node[ground] () {}

			(output) to[short,*-] ++(2,0)
				node[anchor=south] () {$X$}
				to[open,o-o,v^=$v_X$] ++(0,-1.5)
				node[ground] () {};
		\end{circuitikz}
	\end{center}
	For each of the following, replace the input voltage source with the circuit above. For the full circuit, is the value of $v_X$ affected by the attachment of the second block, i.e. does it load the voltage divider?

	\begin{enumerate}
		\item \ \\
		\begin{circuitikz}[scale=0.75, transform shape]
			\draw
			(0,0) node[ground] () {} 
				to[sV,l=$v_\text{in}$] ++(0,1.5)
				to[short] ++(1.5,0)
				to[R=$R_L$] ++(0,-1.5)
				node[ground] () {};
		\end{circuitikz}
		\item \ \\
			\begin{circuitikz}[scale=0.8, transform shape]
	\draw
	(0,0) node[op amp,yscale=-1] (AMP) {}
	(AMP.-) to[short] ++(0,-1) coordinate (bottomLeft)
		to[short] (bottomLeft -| AMP.out)
		to[short] (AMP.out)
		to[short] ++(1,0)
		to[open,o-o,v^=$v_\text{out}$] ++(0,-2)
		node[ground] () {}
	(AMP.+) to[short,-o] ++(-2,0)
		node[anchor=east] () {$X$}
	;
\end{circuitikz}
		\item \ \\
			\begin{circuitikz}[scale=0.8, transform shape]
	\draw
	(0,0) node[op amp,yscale=-1] (AMP) {}
	(AMP.+) to[short,-o] ++(-2,0)
		node[anchor=east] () {$X$}
	(AMP.-) to[short] ++(0,-1.5) coordinate (bottomLeft)
		to[short] (bottomLeft -| AMP.out)
	(AMP.out) to[R,l_=$R_\text{top}$] (bottomLeft -| AMP.out)
		to[R,l_=$R_\text{bottom}$] ++(0,-1.5)
		to[V,l_=$V_\text{REF}$] ++(0,-1.5)
		node[ground] () {}
	(AMP.out) to[short] ++(1,0)
		to[open,o-o,v^=$v_\text{out}$] ++(0,-2)
		node[ground] () {}
	;
\end{circuitikz}
		\item \ \\
			\begin{circuitikz}[scale=0.8, transform shape]
	\draw
	(0,0) node[op amp] (AMP) {}
	(AMP.-) to[short] ++(0,1) coordinate (topLeft)
		to[R,l=$R_f$] (topLeft -| AMP.out)
		to[short] (AMP.out)
		to[short,-o] ++(1,0)
		to[open,o-o,v^=$v_\text{out}$] ++(0,-2)
		node[ground] () {}
	(AMP.-) to[R,l_=$R_s$,-o] ++(-2,0)
		node[anchor=east] () {$X$}
	(AMP.+) to[V,l_=$V_\text{REF}$] ++(0,-2)
		node[ground] () {};
\end{circuitikz}
	\end{enumerate}
	}

\sol{
	What this is asking for can be written in a few equivalent ways:
	\begin{itemize}
		\item Is the Th\'evenin resistance ($R_{\text{in}}$) from the input node of the circuit finite?
		\item Will the circuit in question load the voltage divider above?
		\item Does the circuit draw current from its input node?
	\end{itemize}
	FINISH THIS!!! - this section should have (1) the full circuit analysis (2) the computation finding the input resistance of the circuit.
	\begin{enumerate}
		\item 
		\item
		\item
	\end{enumerate}
}

\ans{
	\begin{enumerate}
		\item This is a voltage buffer! It does not affect the value of $v_x$ of the voltage divider.
		\item This is a noninverting amplifier. Note that no current is drawn from the voltage source because $i_+ = i_- = 0\si{\ampere}$; that said, it does not load the voltage follower and so does not affect the value of $v_X$ from its original value.
		\item This is an inverting amplifier. In this case, current flows across $R_s$, meaning that the amplifier \textit{does} in fact load the voltage divider and affects the value of $v_X$!
	\end{enumerate}
}

\qitem\label{rout}{
	For the following blocks, determine if attaching $R_L$ between the output and ground affects $v_\text{out}$
	\begin{enumerate}
		\item \ \\
			\begin{circuitikz}[scale=0.8, transform shape]
	\draw
	(0,0) node[op amp,yscale=-1] (AMP) {}
	(AMP.-) to[short] ++(0,-1) coordinate (bottomLeft)
		to[short] (bottomLeft -| AMP.out)
		to[short] ++(1.5,0) coordinate (bottomRight)
		to[short] (bottomRight |- AMP.out)
		to[R=$r_o$] (AMP.out)
	(bottomRight |- AMP.out) to[short] ++(1,0)
		to[open,o-o,v^=$v_\text{out}$] ++(0,-2)
		node[ground] () {}
	(AMP.+) to[short] ++(-2,0)
		to[sV,v_=$v_\text{in}$] ++(0,-2)
		node[ground] () {}
	;
\end{circuitikz}
		\item \ \\
			\begin{circuitikz}[scale=0.8, transform shape]
	\draw
	(0,0) node[op amp] (AMP) {}
	(AMP.-) to[short] ++(0,1) coordinate (topLeft)
		to[R,l=$R_f$] (topLeft -| AMP.out)
		to[short] ++(1.5,0) coordinate (topRight)
		to[short] (topRight |- AMP.out)
		to[R=$r_o$] (AMP.out)
	(topRight |- AMP.out) to[short,-o] ++(1,0)
		to[open,o-o,v^=$v_\text{out}$] ++(0,-2)
		node[ground] () {}
	(AMP.-) to[R,l_=$R_s$] ++(-2,0)
		to[sV,v_=$v_\text{in}$] ++(0,-2)
		node[ground] () {}
	(AMP.+) to[V,l_=$V_\text{REF}$] ++(0,-2)
		node[ground] () {};
\end{circuitikz}
		\item \ \\
			\begin{circuitikz}[scale=0.8, transform shape]
	\draw
	(0,0) node[op amp,yscale=-1] (AMP) {}
	(AMP.-) to[short] ++(0,-1) coordinate (bottomLeft)
		to[short] (bottomLeft -| AMP.out)
		to[short] (AMP.out)
		to[R=$r_\ell$] ++(2,0)
		to[open,o-o,v^=$v_\text{out}$] ++(0,-2)
		node[ground] () {}
	(AMP.+) to[short] ++(-2,0)
		to[sV,v_=$v_\text{in}$] ++(0,-2)
		node[ground] () {}
	;
\end{circuitikz}
		\item \ \\
			\begin{circuitikz}[scale=0.8, transform shape]
	\draw
	(0,0) node[op amp] (AMP) {}
	(AMP.-) to[short] ++(0,1) coordinate (topLeft)
		to[R,l=$R_f$] (topLeft -| AMP.out)
		to[short] (AMP.out)
		to[short,-o] ++(1,0)
		to[open,o-o,v^=$v_\text{out}$] ++(0,-2)
		node[ground] () {}
	(AMP.-) to[R,l_=$R_s$] ++(-2,0)
		to[sV,v_=$v_\text{in}$] ++(0,-2)
		node[ground] () {}
	(AMP.+) to[V,l_=$V_\text{REF}$] ++(0,-2)
		node[ground] () {}
	(AMP.out) to[R=$r_\ell$] ++(0,-2)
		node[ground] () {};
\end{circuitikz}
	\end{enumerate}
}

\sol{
	After enough practice, students should be able to do these types of problems by inspection. Poll your class if they think yes/no. If students ask, you should be ready to work out the full solution.}

\ans{
	\begin{enumerate}
		\item The configuration shown in this part is a voltage buffer. Attaching $R_L$ between output and ground won't affect $v_{out}$ because the output voltage remains the same across nodes.
		\item As shown, we essentially have an inverting amplifier for this configuration. However, $r_{\rho}$ is within the output node of the amplifier, attaching $R_L$ will have the same effect as in the previous part. The output voltage still remains the same.
		\item Since we connect $r_o$ outside the output of the amplifier, once we attach $R_L$ between the end node and the ground, we have essentially built a voltage divider with the 2 resistors, which decreases the output voltage!
		\item For this circuit, when we attach $R_L$ between the output and the ground, we can see that $r_l$ and $R_L$ are both in parallel, and the positive end of $v_out$ can be seen directly connected to the output node of the amplifier, which isn't affected by any resistor(s) along the way. Hence, the output voltage won't change.
	\end{enumerate}
}
\end{enumerate}
\empt{\newpage}