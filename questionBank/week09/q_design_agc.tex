% Lydia Lee, lydia.lee@berkeley.edu
\qns{Automatic Gain Control}

\begin{enumerate}
\qitem\label{amp}{
	Design a circuit where $V_\text{out} = -10V_\text{in}$. You are allowed to use
	\begin{itemize}
		\item op amps: $\leq$1. You do not need to specify rail voltages for this subpart.
		\item resistors: as many as you want
		\item capacitors: as many as you want
	\end{itemize}
}
\empt{
	\vspace{1.5cm}
	\begin{circuitikz}
		\draw
		(0,0) to[short] ++(-1,0)
			to[sV,v_=$V_\text{in}$] ++(0,-2)
			node[ground] () {};
	\end{circuitikz}
	\vspace{1.5cm}}

\qitem\label{no_rail}{
	Your design was provided voltage rails at $5\si{\volt}$ and $-5\si{\volt}$, and that your input signal looks like so:
	\begin{center}
		\textcolor{red}{INSERT VOLTAGE FIGURE WITH MAX ABSOLUTE VALUE LESS THAN $0.5\si{\volt}$}
	\end{center}
	Will your signal be faithfully amplified? That is, will your op amp rail/clip?
}
\empt{\vspace{2cm}}

\qitem\label{yes_rail}{
	Again, your design has voltage rails at $\pm 5\si{\volt}$. Your input signal now looks like so:
	\begin{center}
		\textcolor{red}{INSERT VOLTAGE FIGURE WITH MAX ABSOLUTE VALUE AT $1\si{\volt}$}
	\end{center}
	With this input signal, will your amplifier rail/clip?
}
\empt{\vspace{2cm}}

\qitem\label{resistorDAC}{
	Design a circuit which can have a resistance $R$ when an input signal $\phi$ is low and a resistance $nR$ ($n > 1$) when $\phi$ is high. You may use:
	\begin{itemize}
		\item ideal switches: as many as you want
		\item resistors: $\leq 2$
	\end{itemize}
}
\empt{\vspace{4cm}}

\qitem\label{vga}{
	Using your answer from parts \ref{amp} and \ref{resistorDAC}, design a circuit where
	$$V_\text{out} = \begin{cases}
						-5V_\text{in} & \phi = 0\\
						-10V_\text{in} & \phi = V_{DD}
					\end{cases}$$
}
\empt{
	\vspace{2cm}

	\begin{circuitikz}
		\draw
		(0,0) to[short] ++(-1,0)
			to[sV,v_=$V_\text{in}$] ++(0,-2)
			node[ground] () {};
	\end{circuitikz}
	\vspace{2cm}}

\qitem\label{agc}{
	Now let's put it all together. Using
	\begin{itemize}
		\item op amps: as many as you want
		\item switches: as many as you want
		\item resistors: as many as you want
		\item ideal voltage source: as many as you want, not including the provided rails $\pm 5\si{\volt}$
	\end{itemize}
	design a circuit which adjusts the gain setting depending on the input level. You may assume all feedback loops are infinitely fast.
	$$V_\text{out} = \begin{cases}
						-5V_\text{in} & |V_\text{in}| > 0.49\si{\volt}\\
						-10V_\text{in} & |V_\text{in}| < 0.49\si{\volt}
					\end{cases}$$
}
\empt{
	\vspace{2cm}

	\begin{circuitikz}
		\draw
		(0,0) to[short] ++(-1,0)
			to[sV,v_=$V_\text{in}$] ++(0,-2)
			node[ground] () {};
	\end{circuitikz}
	\vspace{2cm}}
\end{enumerate}