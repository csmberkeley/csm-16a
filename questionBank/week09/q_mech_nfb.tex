% Lydia Lee, lydia.lee@berkeley.edu
\qns{Live (And Die) By The Golden Rule(s)}

\sol{
	The purpose of this question is rote practice. While students will have seen many of these topologies, ask students if anyone wants things rederived.}
\begin{enumerate}
\qitem\label{golden_rules}
	What are the ``golden rules'' of op amps? Do any of them have conditions which must hold for them to be true?

	\begin{circuitikz}
		\draw (0,0) node[op amp,yscale=-1] (opamp) { }
		  (-2.5, 0.5) to [short, i=$i_{+}$, o-] (opamp.+) 
		  (-2.5, -0.5) to [short, i=$i_{-}$, o-] (opamp.-) 
		  (opamp.out) node[right ] {$U_{out}$} 
		  ;
		\node[draw=none,text=black] at (-2.8, 0.5) {$u_{+}$};
		\node[draw=none,text=black] at (-2.8, -0.5) {$u_{-}$};

		\draw (0, 0.5) to [short, -o] (0, 1.0);
		\node[draw=none,text=black] at (0, 1.5) {$V_{DD}$};
		\draw (0, -0.5) to [short, -o] (0, -1.0);
		\node[draw=none,text=black] at (0, -1.5) {$V_{SS}$};
	\end{circuitikz}

\ans{
	The Golden Rules of op amps with their necessary preconditions are:
	\begin{itemize}
		\item $i_+ = i_- = 0\si{\ampere}$. This is true of all op amps.
		\item When the op amp is in negative feedback, $u_+ = u_-$
	\end{itemize}}


\qitem\label{amplifier_topologies}
	For each of the following, find $v_\text{out}$ in terms of any input sources and any other given component values. First, solve for when $V_\text{REF} = 0\si{\volt}$. You may work out when $V_\text{REF} \neq 0\si{\volt}$ as a challenge. Assume the voltage rails are sufficient to not distort the output and that all op amps are in negative feedback. 
	\begin{enumerate}
		\item\label{buffer}\ \\
			\begin{circuitikz}[scale=0.8, transform shape]
	\draw
	(0,0) node[op amp,yscale=-1] (AMP) {}
	(AMP.-) to[short] ++(0,-1) coordinate (bottomLeft)
		to[short] (bottomLeft -| AMP.out)
		to[short] (AMP.out)
		to[short] ++(1,0)
		to[open,o-o,v^=$v_\text{out}$] ++(0,-2)
		node[ground] () {}
	(AMP.+) to[short,-o] ++(-2,0)
		node[anchor=east] () {$X$}
	;
\end{circuitikz}

			\sol{
				\begin{align*}
					v_\text{out} &= v_-\\
						&= v_+\\
						&= v_\text{in}
				\end{align*}}

			\ans{
				$v_\text{out} = v_\text{in}$}
		
		\item\label{buffer_res}\ \\
			\input{../../questionBank/week09/q_mech_nfb_figs/buffer_res}
			
			\sol{
				\begin{align*}
					v_\text{out} &= v_-\\
						&= v_+\\
						&= v_\text{in}
				\end{align*}}

			\ans{
				$v_\text{out} = v_\text{in}$. Remember that no current flows into the input terminals of the op amp!}

		\item\label{noninverting_amp}\ \\
			\begin{circuitikz}[scale=0.8, transform shape]
	\draw
	(0,0) node[op amp,yscale=-1] (AMP) {}
	(AMP.+) to[short,-o] ++(-2,0)
		node[anchor=east] () {$X$}
	(AMP.-) to[short] ++(0,-1.5) coordinate (bottomLeft)
		to[short] (bottomLeft -| AMP.out)
	(AMP.out) to[R,l_=$R_\text{top}$] (bottomLeft -| AMP.out)
		to[R,l_=$R_\text{bottom}$] ++(0,-1.5)
		to[V,l_=$V_\text{REF}$] ++(0,-1.5)
		node[ground] () {}
	(AMP.out) to[short] ++(1,0)
		to[open,o-o,v^=$v_\text{out}$] ++(0,-2)
		node[ground] () {}
	;
\end{circuitikz}

			\sol{
				\begin{align*}
					\frac{v_\text{out}-v_\text{in}}{R_\text{top}} &= \frac{v_\text{in}-V_\text{REF}}{R_\text{bottom}}\\
					v_\text{out} &= v_\text{in} + \frac{R_\text{top}}{R_\text{bottom}}\left(v_\text{in}-V_\text{REF}\right)\\
						&= v_\text{in}\left(1 + \frac{R_\text{top}}{R_\text{bottom}}\right) - V_\text{REF}\left(\frac{R_\text{top}}{R_\text{bottom}}\right)
				\end{align*}}

			\ans{
				$v_\text{out} = v_\text{in}\left(1 + \frac{R_\text{top}}{R_\text{bottom}}\right) - V_\text{REF}\left(\frac{R_\text{top}}{R_\text{bottom}}\right)$}

		\item\label{inverting_amp}\textit{Hint: Is this similar at all to part \ref{noninverting_amp}?}\\
			\begin{circuitikz}[scale=0.8, transform shape]
	\draw
	(0,0) node[op amp] (AMP) {}
	(AMP.-) to[short] ++(0,1) coordinate (topLeft)
		to[R,l=$R_f$] (topLeft -| AMP.out)
		to[short] (AMP.out)
		to[short,-o] ++(1,0)
		to[open,o-o,v^=$v_\text{out}$] ++(0,-2)
		node[ground] () {}
	(AMP.-) to[R,l_=$R_s$,-o] ++(-2,0)
		node[anchor=east] () {$X$}
	(AMP.+) to[V,l_=$V_\text{REF}$] ++(0,-2)
		node[ground] () {};
\end{circuitikz}

			\sol{
				\begin{align*}
					\frac{v_\text{in}-V_\text{REF}}{R_s} &= \frac{V_\text{REF}-v_\text{out}}{R_f}\\
					V_\text{REF}-v_\text{out} &= \frac{R_f}{R_s}\left(v_\text{in}-V_\text{REF}\right)\\
					v_\text{out} &= V_\text{REF}\left(1 + \frac{R_f}{R_s}\right) - v_\text{in}\left(\frac{R_f}{R_s}\right)
				\end{align*}}

			\ans{
				$v_\text{out} = v_\text{in}\left(-\frac{R_f}{R_s}\right) + V_\text{REF}\left(\frac{R_f}{R_s} + 1\right)$. Note that this is exactly the same as \ref{noninverting_amp} with $v_\text{in}$ and $V_\text{REF}$ switched!}

		\item\label{inverting_summing_amp}\ \\
			\begin{circuitikz}[scale=0.8, transform shape]
	\draw
	(0,0) node[op amp] (AMP) {}
	(AMP.-) to[short] ++(0,1) coordinate (topLeft)
		to[R,l=$R_f$] (topLeft -| AMP.out)
		to[short] (AMP.out)
		to[short,-o] ++(1,0)
		to[open,o-o,v^=$v_\text{out}$] ++(0,-2)
		node[ground] () {}
	(AMP.-) to[short] ++(-.5,0) coordinate (summing)
		to[R,l_=$R_{s1}$] ++(-2,0)
		to[/tikz/circuitikz/bipoles/length=1cm,sV,v_=$v_\text{in1}$] ++(0,-1)
		node[ground] () {}
	(summing) to[short] ++(0,2)
		to[R,l_=$R_{s2}$] ++(-2,0)
		to[/tikz/circuitikz/bipoles/length=1cm,sV,v_=$v_\text{in2}$] ++(0,-1)
		node[ground] () {}
	(AMP.+) to[V,l_=$V_\text{REF}$] ++(0,-2)
		node[ground] () {};
\end{circuitikz}
			
			\sol{
				\textbf{KCL}: There are a few ways to do this, but the fastest way is through KCL, a.k.a. nodal analysis
				\begin{align*}
					\frac{v_\text{in1}-V_\text{REF}}{R_{s1}} + \frac{v_\text{in2}-V_\text{REF}}{R_{s2}} &= \frac{V_\text{REF}-v_\text{out}}{R_f}\\
					V_\text{REF}-v_\text{out} &= R_f\left(\frac{v_\text{in1}}{R_{s1}} + \frac{v_\text{in2}}{R_{s2}} - V_\text{REF}\left(\frac{1}{R_{s1}} + \frac{1}{R_{s2}}\right)\right)\\
					v_\text{out} &= v_\text{in1}\left(\frac{-R_f}{R_{s1}}\right) + v_\text{in2}\left(\frac{-R_f}{R_{s2}}\right) + V_\text{REF}\left(\frac{R_f}{R_{s1}} + \frac{R_f}{R_{s2}} + 1\right)
				\end{align*}
				\textbf{Superposition}: As with all circuit problems, this can also be solved via superposition. Turning off all independent sources but $v_\text{in1}$, $v_\text{in2}$, and $V_\text{REF}$ in order:
				\begin{align*}
					\frac{v_\text{in1}}{R_{s1}} &= \frac{0-v_\text{out1}}{R_f}\\
					v_\text{out1} &= v_\text{in1}\left(-\frac{R_f}{R_{s1}}\right)\\
				\end{align*}
				Finding the effect of $v_\text{in2}$ is identical to the process used for $v_\text{in1}$
				\begin{align*}
					\frac{v_\text{in2}}{R_{s2}} &= \frac{0-v_\text{out2}}{R_f}\\
					v_\text{out2} &= v_\text{in2}\left(-\frac{R_f}{R_{s2}}\right)\\
				\end{align*}
				With just $V_\text{REF}$ on and the other two independent voltages sources turned off, we see a noninverting amplifier.
				\begin{align*}
					V_\text{REF} &= v_\text{out,REF}\frac{R_{s1}||R_{s2}}{R_f + R_{s1}||R_{s2}}\\
					v_\text{out,REF} &= V_\text{REF}\left(1 + \frac{R_f}{R_{s1}||R_{s2}}\right)
				\end{align*}
				And adding all of the results together:
				\begin{align*}
					v_\text{out} &= v_\text{out1} + v_\text{out2} + v_\text{outREF}\\
						&= v_\text{in1}\left(-\frac{R_f}{R_{s1}}\right) + v_\text{in2}\left(-\frac{R_f}{R_{s2}}\right) + V_\text{REF}\left(1 + \frac{R_f}{R_{s1}||R_{s2}}\right)
				\end{align*}}

			\ans{
				\begin{align*}
					v_\text{out} &= v_\text{in1}\left(\frac{-R_f}{R_{s1}}\right) + v_\text{in2}\left(\frac{-R_f}{R_{s2}}\right) + V_\text{REF}\left(\frac{R_f}{R_{s1}} + \frac{R_f}{R_{s2}} + 1\right)\\
						&= v_\text{in1}\left(-\frac{R_f}{R_{s1}}\right) + v_\text{in2}\left(-\frac{R_f}{R_{s2}}\right) + V_\text{REF}\left(1 + \frac{R_f}{R_{s1}||R_{s2}}\right)
				\end{align*}}

		\item\label{transimpedance_amp}\ \\
			\input{../../questionBank/week09/q_mech_nfb_figs/transimpedance_amp}

			\sol{
				\begin{align*}
					i_\text{in} &= \frac{V_\text{REF}-v_\text{out}}{R}\\
					v_\text{out} &= V_\text{REF}-i_\text{in}R
				\end{align*}}

			\ans{
				$v_\text{out} = i_\text{in}(-R) + V_\text{REF}$}
		
		\item\label{integrator_voltage}Assume you know $v_\text{out}(t_0)$.\\
			\begin{circuitikz}[scale=0.8, transform shape]
	\draw
	(0,0) node[op amp] (AMP) {}
	(AMP.out) to[short] ++(0,2) coordinate (topRight)
		to[C,l_=$C$] (topRight -| AMP.-)
		to[short] (AMP.-)
		to[R,l_=$R$] ++(-2,0)
		to[sV,v_=$v_\text{in}$] ++(0,-2) coordinate (groundPlane)
		node[ground] () {}
	(AMP.+) to[short] ++(-.5,0) coordinate
		to[V,v=$V_\text{REF}$] ++(0,-2)
		node[ground] () {}
	(AMP.out) to[short] ++(1,0) coordinate (outOut)
		to[open,o-o,v^=$v_\text{out}$] (outOut |- groundPlane)
		node[ground] () {};
\end{circuitikz}
			
			\sol{
				\begin{align*}
					\frac{v_\text{in}-V_\text{REF}}{R} &= C\frac{d(V_\text{REF}-v_\text{out})}{dt}\\
						&= C\left(\frac{dV_\text{REF}}{dt} - \frac{dv_\text{out}}{dt}\right)\\
						&= -C\frac{dv_\text{out}}{dt} \longleftarrow \frac{dV_\text{REF}}{dt} = 0\\
					\frac{dv_\text{out}}{dt} &= -\frac{1}{RC}(v_\text{in}-V_\text{REF})\\
					v_\text{out}(t) - v_\text{out}(t_0) &= \int_{t_0}^t-\frac{v_\text{in}-V_\text{REF}}{RC}d\tau\\
					v_\text{out}(t) &= v_\text{out}(t_0) + \int_{t_0}^t\frac{V_\text{REF}-v_\text{in}}{RC}d\tau
				\end{align*}}

			\ans{
				$v_\text{out}(t) = v_\text{out}(t_0) + \int_{t_0}^t\frac{V_\text{REF}-v_\text{in}}{RC}d\tau$. Note that your voltages may not be constant with respect to time!}
		\empt{\newpage}
		\item\label{differentiator}\textbf{(PRACTICE)}\\
			\begin{circuitikz}[scale=0.8, transform shape]
	\draw (0,0) node[op amp] (AMP) {}
        (AMP.out) to[short] ++(0,2) coordinate (topRight)
        to[R,l_=$R$] (topRight -| AMP.-)
        to[short] (AMP.-)
        to[C,l_=$C$] ++(-2,0)
        to[sV,v_=$v_\text{in}$] ++(0,-2)
        node[ground] () {}
    (AMP.+) to[short] ++(-.5,0)
        to[V,v=$V_\text{REF}$] ++(0,-2)
        node[ground] () {}
    (AMP.out) to[short] ++(1,0)
        to[open,v^=$v_\text{out}$,o-o] ++(0,-2)
        node[ground] () {};
\end{circuitikz}

			\sol{
				\begin{align*}
					\frac{V_\text{REF}-v_\text{out}}{R} &= C\frac{d(v_\text{in}-V_\text{REF})}{dt}\\
						&= C\left(\frac{d(v_\text{in}}{dt}-\frac{V_\text{REF})}{dt}\right)\\
					v_\text{out} &= V_\text{REF} - RC\frac{dv_\text{in}}{dt} \longleftarrow \frac{dV_\text{REF}}{dt} = 0
				\end{align*}}

			\ans{
				$v_\text{out} = V_\text{REF} - RC\frac{dv_\text{in}}{dt}$}
		
		% \item\label{instrumentation_amp}\textbf{(CHALLENGE PRACTICE)}\\
		% 	\begin{circuitikz}[scale=0.8, transform shape]
	\draw
	(0,0) node[op amp,yscale=-1] (AMP_TOP) {}
	(0,-6) node[op amp] (AMP_BOT) {}
	(5,-3) node[op amp] (AMP_OUT) {}

	% Input voltage sources
	(AMP_TOP.+) to[short] ++(-1,0)
		to[sV,v_=$v_\text{in1}$] ++(0,-2)
		node[ground] () {}

	(AMP_BOT.+) to[short] ++(-1,0)
		to[sV,v_=$v_\text{in2}$] ++(0,-2)
		node[ground] () {}

	% First layer of resistors
	(AMP_TOP.-) to[short] ++(0,-1.5) coordinate (topMinusExtension)
		to[short] (topMinusExtension -| AMP_TOP.out)
		to[R=$R_1$] (AMP_TOP.out)
	(AMP_BOT.-) to[short] ++(0,1.5) coordinate (botMinusExtension)
		to[short] (botMinusExtension -| AMP_BOT.out)
		to[R=$R_1$] (AMP_BOT.out)
	(topMinusExtension -| AMP_TOP.out) to[R=$R_{\text{gain}}$] (botMinusExtension -| AMP_BOT.out)

	% Second layer of resistors
	(AMP_TOP.out) to[R=$R_2$] (AMP_TOP.out -| AMP_OUT.-)
		to[short] (AMP_OUT.-)
	(AMP_BOT.out) to[R=$R_2$] (AMP_BOT.out -| AMP_OUT.+)
		to[short] (AMP_OUT.+)

	% Third layer of resistors
	(AMP_OUT.out) to[short] (AMP_OUT.out |- AMP_TOP.out)
		to[R=$R_3$] (AMP_TOP.out -| AMP_OUT.-)
		to[short] (AMP_OUT.-)
	(AMP_OUT.+) to[short] (AMP_OUT.+ |- AMP_BOT.out)
		to[R=$R_3$] (AMP_BOT.out -| AMP_OUT.out)
		node[ground] () {}
	
	% Final output
	(AMP_OUT.out) to[short] ++(1,0)
		to[open,o-o,v^=$v_\text{out}$] ++(0,-3)
		node[ground] () {}
	;
\end{circuitikz}
	\end{enumerate}


\end{enumerate}