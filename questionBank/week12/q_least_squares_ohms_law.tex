% Lydia Lee, lydia.lee@berkeley.edu

\qns{And You Thought You Could Ignore Circuits Until Dead Week}

\begin{enumerate}

\qitem\label{ohmsLaw}{
	Write Ohm's Law for a resistor.}

\ans{
	For the resistor
	\begin{center}
		\begin{circuitikz}
			\draw
			(0,0) to[R,l=$R$,v=$V_R$,i=$I_R$] ++(2,0);
		\end{circuitikz}
	\end{center}
	$$V_R = I_RR$$}

\qitem\label{singlePoint}{
	You're given the following test setup and told to find $R_{Th}$ between the two terminals of the mystery box. What is $R_{Th}$ of the mystery box between the two terminals in terms of $V_S$ and $I_\text{measure}$?
	\begin{center}
		\begin{circuitikz} [baseline=(current bounding box.center)]
	\ctikzset { label/align = straight }
	\draw (0,0)
		to[V=$V_{S}$,invert] (0,2)
		to[short, i=$I_\text{measure}$, -o] (4,2)
		to[short] (4.5,2)
		(0,0) to[short,-o] (4,0)
		to[short] (4.5,0);
	\node[draw,minimum width=2cm,minimum height=2.4cm,anchor=south west] at (4.5,-0.2){Mystery Box};
\end{circuitikz}
	\end{center}}

\ans{
	\begin{align*}
		R_{Th} &= \frac{V_S}{I_\text{measure}}\\
		R_{Th} &= \frac{V_S}{I_\text{measure}}
	\end{align*}}

\qitem\label{twoNoisyMeasurements}{
	You think you've figured out how to find $R_{Th}$! You've taken the following measurements:
	\begin{center}
		\begin{tabular}{|c|c|c|}
			\hline
			Measurement \# & $V_S$ & $I_\text{measure}$\\\hline
			1 & $1\si{\volt}$ & $1\si{\ampere}$\\\hline
			2 & 
		\end{tabular}
	\end{center}
	Using the information above, formulate a least squares problem whose answer provides an estimate of $R_{Th}$.
	}

\end{enumerate}