% Authors: Aditya Baradwaj, Lydia Lee
% Emails: abaradwaj@berkeley.edu, lydia.lee@berkeley.edu

\qns{Trilateration}\\
Suppose you want to write the backend for a totally not sinister locationing app. More specifically, you want to be able to determine the user's position $\vec{x} \in \mathbb{R}^2$ using cell tower information. You know the following:
\begin{itemize}
	\item The position of the cell towers: $\vec{a_1}$, $\vec{a_2}$, and $\vec{a_3} \in \mathbb{R}^2$
	\item The absolute distance of the user's cell phone relative to the three towers: $d_1$, $d_2$, and $d_3$.
\end{itemize}
\begin{center}
	\begin{tikzpicture}[scale=1.3]
	
	%% coordinates
	\coordinate (O) at (0,0);
	\coordinate (a1) at (130:1.5);
	\coordinate (a2) at (20:3);
	\coordinate (a3) at (-70:2);
	
	%% lines
	\draw[black,thick, -] (O) -- (a1);
	\draw[black,thick, -] (O) -- (a2);
	\draw[black,thick, -] (O) -- (a3);
	
	%% point labels
	\node[] at (130:1.8)  {$\vec{a}_1$};
	\node[] at (20:3.3)  {$\vec{a}_2$};
	\node[] at (-70:2.3)  {$\vec{a}_3$};
	\node[] at (-20:0.35)  {$\vec{x}$};
	
	%% distances
	\node[] at (110:0.8)  {$d_1$};
	\node[] at (8:1.6)  {$d_2$};
	\node[] at (-87:1.1)  {$d_3$};
	
	%% center dot
	\draw[fill] (O) circle [radius=0.1];
	\draw[fill] (a1) circle [radius=0.05];
	\draw[fill] (a2) circle [radius=0.05];
	\draw[fill] (a3) circle [radius=0.05];
	\end{tikzpicture}
\end{center}

\begin{enumerate}
\qitem{Write $d_i$ in terms of $\vec{x}$ and $\vec{a_i}$ where $i=1, 2, 3$.}

\ans{
$$\norm{\vec{x} - \vec{a_1}}_2 = d_1$$
$$\norm{\vec{x} - \vec{a_2}}_2 = d_2$$
$$\norm{\vec{x} - \vec{a_3}}_2 = d_3$$
These equations relate the position of the user to the distance between the user and each tower.
}

\qitem{
	Write these equations in terms of inner products of vectors. Are these equations affine (linear + a constant) with respect to $\vec{x}$? \textit{Hint: $\norm{\vec{v}}_2^2 = \vec{v}^T\vec{v}$}}

\ans{
By squaring each side, we get the following:
$$(\vec{x} - \vec{a_1})^T(\vec{x} - \vec{a_1}) = d_1^2$$
$$(\vec{x} - \vec{a_2})^T(\vec{x} - \vec{a_2}) = d_2^2$$
$$(\vec{x} - \vec{a_3})^T(\vec{x} - \vec{a_3}) = d_3^2$$

To be affine in $\vec{x}$, our equations should follow the form
$$A\vec{x}+\vec{c} = \vec{0}$$ where $A$, $\vec{c}$ are independent of $\vec{x}$. The use of matrices and vectors is for generality; $\vec{c}$ can be a constant just as $A$ can be a row vector.

Expanding the LHS, we get:
$$\vec{x}^T\vec{x} - 2\vec{a_1}^T\vec{x} + \vec{a_1}^T\vec{a_1} = d_1^2$$
$$\vec{x}^T\vec{x} - 2\vec{a_2}^T\vec{x} + \vec{a_2}^T\vec{a_2} = d_2^2$$
$$\vec{x}^T\vec{x} - 2\vec{a_3}^T\vec{x} + \vec{a_3}^T\vec{a_3} = d_3^2$$

These equations are \textit{not} affine in $\vec{x}$ because of the $\vec{x}^T\vec{x}$ term.
}

\qitem{
Use these nonlinear equations to obtain equations which are affine with respect to $\vec{x}$.
}

\ans{
For clarity we'll repeat the equations from above:
\begin{align}
	d_1^2 &= \vec{x}^T\vec{x} - 2\vec{a_1}^T\vec{x} + \vec{a_1}^T\vec{a_1}\label{eqn1}\\
	d_2^2 &= \vec{x}^T\vec{x} - 2\vec{a_2}^T\vec{x} + \vec{a_2}^T\vec{a_2}\label{eqn2}\\
	d_3^3 &= \vec{x}^T\vec{x} - 2\vec{a_3}^T\vec{x} + \vec{a_3}^T\vec{a_3}\label{eqn3}
\end{align}

To be affine in $\vec{x}$, we need to get rid of the $\vec{x}^T\vec{x}$ term, so we subtract \ref{eqn1} from the other two to eliminate the $\vec{x}^T\vec{x}$ term.

$$d_2^2 - d_1^2 = 2(\vec{a_1}^T\vec{x} - \vec{a_2}^T\vec{x}) + (\vec{a_2}^T\vec{a_2} - \vec{a_1}^T\vec{a_1})$$
$$d_3^2 - d_1^2 = 2(\vec{a_1}^T\vec{x} - \vec{a_3}^T\vec{x}) + (\vec{a_3}^T\vec{a_3} - \vec{a_1}^T\vec{a_1})$$

With a bit of massaging to group $\vec{x}$ together:
$$2(\vec{a_1} - \vec{a_2})^T\vec{x} = \vec{a_1}^T\vec{a_1} - \vec{a_2}^T\vec{a_2} + d_2^2 - d_1^2$$
$$2(\vec{a_1} - \vec{a_3})^T\vec{x} = \vec{a_1}^T\vec{a_1} - \vec{a_3}^T\vec{a_3} + d_3^2 - d_1^2$$

Or, in matrix-vector form,

$$2 \begin{bmatrix}
(\vec{a_1} - \vec{a_2})^T \\
(\vec{a_1} - \vec{a_3})^T
\end{bmatrix} \vec{x}
= \begin{bmatrix}
\norm{a_1}^2 - \norm{a_2}^2 + d_2^2 - d_1^2 \\
\norm{a_1}^2 - \norm{a_2}^2 + d_3^2 - d_1^2
\end{bmatrix}$$
}

\qitem{
Suppose now that instead of cell towers which are (basically) coplanar with the user, you're using satellites. In other words, now $\vec{x}, \vec{a_i} \in \mathbb{R}^3$. Would this system of equations be sufficient to find $\vec{x}\in\mathbb{R}^3$? If not, then how many satellites do you need to locate the user?}

\ans{
No, this is not sufficient. We have only 2 equations, but 3 variables. So, this system is necessarily underdetermined. If instead we had 4 satellites, then by subtracting one equation from all the rest, we would have 3 equations, and we could then solve the system.
}

\qitem{
In real life, we want to not only triangulate the user's position, but also keep track of this as it changes over time. So, we are effectively solving for a 4-dimensional $\vec{x}$, where the 4th dimension is time. In this situation, how many satellites would you need?
}

\ans{
You would need 5 satellites to give you 4 equations. These would have to be distinct satellites, otherwise the equations would be linearly dependent.
}

\end{enumerate}