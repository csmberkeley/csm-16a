% Author: Mudit Gupta and Paul Shao
% Email: mudit+csm16a@berkeley.edu
% Email: paulshaoyuqiao1@berkeley.edu

\qns{First Proof}

\sol{Prereq: Knowing what linear independence and dependence are \\
Description: A very simple and basic proof about linear independence.}


Prove that a subset of a finite linear independent set of vectors is linearly independent.

\sol{This is probably pretty early for when students will see proof. Very carefully introduce general proving techniques. Take the question, write down what is given in mathematical notation, and write out what needs to be proven in mathematical notation. A proof is essentially going from the 'given' to the 'to prove'. \\
Another note is that remember to assume that students have not taken CS70. Assume that they do not know proof techniques such as proof by contradiction, direct proof, induction, etc. This question is a proof by contradiction, so introduce it as such. \\
Proof by contradiction is  not taught in 16A, so it is a good idea to go over the general structure of a proof in this format - assuming the negation of the statement you are trying to prove, and then using reductions to show an impossible scenario/contradiction.
Final note: explain the 'without loss of generality' in the 'To Prove' section. Why }

\ans{
    \\
    This problem can be tackled from two different approaches: \textbf{Direct Proof} and \textbf{Proof By Contradiction}. If you are not sure what the second approach means, we have also provided an explanation at the beginning of the second approach. \\ \\
    \textbf{First Approach: }\\
    The problem can be translated mathematically to be the following: \\
    Given a set of $n$ vectors $N = \{\vec{v_1}, \vec{v_2}, \dots \vec{v_n}\}$ that are linearly independent, we want to show that any subset of vectors in $N$ is also linearly independent. \\
    Since $N$ is a linearly independent set, we know that if there exists a set of constants $c_1, c_2, \dots, c_n$ such that:
    $$c_1 \vec{v_1} + c_2 \vec{v_2} + \dots + c_n \vec{v_n} = 0$$
    then it must be true that $c_1 = c_2 = \dots = c_n = 0$. \\
    To also symbolically represent a subset of $N$, since we can shuffle the array or arrange the vectors as much as we want, the following form generalizes to any susbet of vectors in $N$: \\
    We want to show that for some $k \leq n$, $\{\vec{v_1}, \vec{v_2}, \dots \vec{v_k}\}$ is also a set of linearly independent vectors. \\
    Following the same setup based on the definition of linear independence, suppose there exists a set of constants $b_1, b_2, \dots, b_k$ such that:
    $$b_1 \vec{v_1} + b_2 \vec{v_2} + \dots + b_k \vec{v_k} = 0$$
    Now, we can add more "zeros" to both sides of the equations such that we are extending the linear combinations all the way from $v_k$ to $v_n$:
    $$b_1 \vec{v_1} + b_2 \vec{v_2} + \dots + b_k \vec{v_k} + (0)\vec{v_{k+1}} + \dots + (0)\vec{v_n} = 0$$
    Now, since we know that the set $N = \{\vec{v_1}, \vec{v_2}, \dots \vec{v_n}\}$ is linearly independent, we know that all coefficients of the given vectors must be equal to 0 (per definition of linear independence). This means that $b_1 = b_2 = \dots = b_k = 0$ as well. \\
    Hence, it follows that $\{\vec{v_1}, \vec{v_2}, \dots \vec{v_k}\}$ is also a set of linearly independent vectors. \\ \\
    \textbf{Second Approach: } \\
    This problem can also be approached from another direction using a technique called \textbf{Proof by Contradiction}. \\ \\
    \textbf{More Explanation on Proof by Contradiction:} In essence, the technique assumes the opposite of what we are trying to prove (so if the property we are proving is called $P$, we want to prove \textit{not} $P$ is true), and then reaches two \textbf{mutually contradictory} assertions (statements) (i.e., Property $A$ is true and also false at the same time). Since both statements can't be simultaneously true, this leads us to conclude the property not $P$ is in fact wrong, so $P$ must be true. \\ \\
	$\textbf{Given:}$
	$\vec{v}_1, \vec{v}_2, \ldots, \vec{v}_n$ are linearly independent. This, by definition of linear independence, means that if there exist $\alpha_1, \alpha_2, \ldots, \alpha_n$, such that:
	 $$\alpha_1\vec{v}_1 + \alpha_2\vec{v}_2 + \ldots + \alpha_n\vec{v}_n = 0$$ then  $$\alpha_1=\alpha_2=\ldots=\alpha_n=0$$ 
	 In other words, the only solution to the above $\alpha$s is that the $\alpha$s are all $0$.\\
	$\textbf{To Prove:}\ \beta_1\vec{v}_1 + \beta_2\vec{v}_2 + \ldots + \beta_k\vec{v}_k = 0 \implies \beta_1=\beta_2=\ldots=\beta_k=0$. \\ Note that $\vec{v}_1, \vec{v}_2, \ldots, \vec{v}_k$ are a subset of $\vec{v}_1, \vec{v}_2, \ldots, \vec{v}_n$.

	Assume that $ \beta_1\vec{v}_1 + \beta_2\vec{v}_2 + \ldots + \beta_k\vec{v}_k = 0 $ is true but not $\beta_1=\beta_2=\ldots=\beta_k=0$. 

	Consider $\beta_1\vec{v}_1 + \beta_2\vec{v}_2 + \ldots + \beta_k\vec{v}_k + 0\vec{v}_{k+1} + 0\vec{v}_{k+2} + \ldots + 0\vec{v}_n$. If $\beta_1\vec{v}_1 + \beta_2\vec{v}_2 + \ldots + \beta_k\vec{v}_k = 0$ then 

	$$\beta_1\vec{v}_1 + \beta_2\vec{v}_2 + \ldots + \beta_k\vec{v}_k + 0\vec{v}_{k+1} + 0\vec{v}_{k+2} + \ldots + 0\vec{v}_n = 0$$ 

	However, since we assumed that not all $\beta_1, \beta_2, \ldots, \beta_k$ are 0, this means that the set $\{\vec{v}_1, \vec{v}_2, \ldots, \vec{v}_n\}$ is not linearly independent, which is a contradiction because it is given that the set is linearly independent. Therefore, $\beta_1=\beta_2=\ldots=\beta_k=0$ must have been true. 
}