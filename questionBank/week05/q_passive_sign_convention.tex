% Lydia Lee, Spring 2019
% lydia.lee@berkeley.edu
\qns{Passive (Aggressive) Sign Convention}

For the following components, label the $V_\text{element}$ or $I_\text{element}$ given the $I_\text{element}$ or $V_\text{element}$, respectively. \textit{Hint: The value of the voltage and current sources shouldn't affect passive sign convention---remember that voltage and current can be negative!}
\begin{enumerate}

\qitem\label{ques:blackBox_voltage}{\ \\
\begin{center}
	\begin{circuitikz}[scale=0.75]
		\ctikzset{resistor = european}
		\draw (0,0) to[R, v=$V_\text{element}$] ++(4,0);
	\end{circuitikz}
\end{center}
}
\ans{
\begin{center}
	\begin{circuitikz}[scale=0.75]
		\ctikzset{resistor = european}
		\draw (0,0) to[R, v=$V_\text{element}$, i=$I_\text{element}$] ++(4,0);
	\end{circuitikz}
\end{center}
}

\qitem\label{ques:blackBox_current}{\ \\
\begin{center}
	\begin{circuitikz}[scale=0.75]
		\ctikzset{resistor = european}
		\draw (0,0) to[R, i=$I_\text{element}$] ++(4,0);
	\end{circuitikz}
\end{center}
}
\ans{
\begin{center}
	\begin{circuitikz}[scale=0.75]
		\ctikzset{resistor = european}
		\draw (0,0) to[R, v=$V_\text{element}$, i=$I_\text{element}$] ++(4,0);
	\end{circuitikz}
\end{center}
}

\qitem\label{ques:vSrc_voltage_forward}{\ \\
\begin{center}
	\begin{circuitikz}[scale=0.75]
		\draw 
		(0,0) to[V, v=$V_\text{S}$, invert] ++(0,3)
			to[open] ++(.5,0)
			to[open, v^=$V_\text{element}$] ++(0,-3);
	\end{circuitikz}
\end{center}
}
\ans{
\begin{center}
	\begin{circuitikz}[scale=0.75]
		\draw 
		(0,0) to[V, v=$V_\text{S}$, i=$I_\text{element}$, invert] ++(0,3)
			to[open] ++(.5,0)
			to[open, v^=$V_\text{element}$] ++(0,-3);
	\end{circuitikz}
\end{center}
}

\qitem\label{ques:vSrc_voltage_backwards}{\ \\
\begin{center}
	\begin{circuitikz}[scale=0.75]
		\draw 
		(0,0) to[V, v=$1\si{\volt}$, invert] ++(0,3)
		(0,0) to[open] ++(.5,0)
			to[open, v_=$V_\text{element}$] ++(0,3);
	\end{circuitikz}
\end{center}
}

\sol{If students are confused by this answer, draw the source on the board, then a box around it. Shade in the box so the voltage source is obscured, and now the problem is identical to part \ref{ques:blackBox_voltage}. This particular style of question has featured on several exams, and it's good to have lots of practice with this when calculating power.
}

\ans{
\begin{center}
	\begin{circuitikz}[scale=0.75]
		\draw 
		(0,0) to[V, v=$1\si{\volt}$, invert] ++(0,3)
		(0,0) to[open] ++(.5,0)
			to[open, v_=$V_\text{element}$] ++(0,3)
		(0,0) to[open, i=$I_\text{element}$, invert] ++(0,3);
	\end{circuitikz}
\end{center}
}

\qitem\label{ques:vSrc_current_forward}{\ \\
\begin{center}
	\begin{circuitikz}[scale=0.75]
		\draw 
		(0,0) to[V, v=$V_\text{S}$, i=$I_\text{element}$, invert] ++(0,3);
	\end{circuitikz}
\end{center}
}
\ans{
\begin{center}
	\begin{circuitikz}[scale=0.75]
		\draw 
		(0,0) to[V, v=$V_\text{S}$, i=$I_\text{element}$, invert] ++(0,3)
			to[open] ++(.5,0)
			to[open, v^=$V_\text{element}$] ++(0,-3);
	\end{circuitikz}
\end{center}
}

\newpage
\qitem\label{ques:vSrc_current_backwards}{\ \\
\begin{center}
	\begin{circuitikz}[scale=0.75]
		\draw 
		(0,0) to[V, v=$-1\si{\volt}$, invert] ++(0,3)
		(0,0) to[open, i=$I_\text{element}$, invert] ++(0,3);
	\end{circuitikz}
\end{center}
}
\ans{
\begin{center}
	\begin{circuitikz}[scale=0.75]
		\draw 
		(0,0) to[V, v=$-1\si{\volt}$, invert] ++(0,3)
		(0,0) to[open] ++(.5,0)
			to[open, v_=$V_\text{element}$] ++(0,3)
		(0,0) to[open, i=$I_\text{element}$, invert] ++(0,3);
	\end{circuitikz}
\end{center}
}

\qitem\label{ques:iSrc_voltage_forward}{\textbf{(PRACTICE)}

\begin{center}
	\begin{circuitikz}[scale=0.75]
		\draw 
		(0,0) to[I, l=$I_\text{S}$, invert] ++(0,3)
			to[open] ++(.5,0)
			to[open, v^=$V_\text{element}$] ++(0,-3);
	\end{circuitikz}
\end{center}
}
\ans{
\begin{center}
	\begin{circuitikz}[scale=0.75]
		\draw 
		(0,0) to[I, l=$I_\text{S}$, invert] ++(0,3)
			to[open] ++(.5,0)
			to[open, v^=$V_\text{element}$] ++(0,-3)
		(0,3) to[open, i^=$I_\text{element}$] ++(0,-3);
	\end{circuitikz}
\end{center}
}

\qitem\label{ques:iSrc_voltage_backwards}{\textbf{(PRACTICE)}

\begin{center}
	\begin{circuitikz}[scale=0.75]
		\draw 
		(0,0) to[I, l=$I_\text{S}$, invert] ++(0,3)
		(0,0) to[open] ++(.5,0)
			to[open, v_=$V_\text{element}$] ++(0,3);
	\end{circuitikz}
\end{center}
}
\ans{
\begin{center}
	\begin{circuitikz}[scale=0.75]
		\draw 
		(0,0) to[I, l=$I_\text{S}$, invert] ++(0,3)			
		(.5,0) to[open, v_=$V_\text{element}$] ++(0,3)
		(0,0) to[open, i^=$I_\text{element}$] ++(0,3);
	\end{circuitikz}
\end{center}
}

\qitem\label{ques:iSrc_current_forward}{\textbf{(PRACTICE)}

\begin{center}
	\begin{circuitikz}[scale=0.75]
		\draw 
		(0,0) to[I, l=$I_\text{S}$, invert] ++(0,3)
			to[open, i=$I_\text{element}$] ++(0,-3);
	\end{circuitikz}
\end{center}
}
\ans{
\begin{center}
	\begin{circuitikz}[scale=0.75]
		\draw 
		(0,0) to[I, l=$I_\text{S}$, invert] ++(0,3)
			to[open, i=$I_\text{element}$] ++(0,-3)
		(.5,3) to[open, v^=$V_\text{element}$] ++(0,-3);
	\end{circuitikz}
\end{center}
}

\qitem\label{ques:iSrc_current_backwards}{\textbf{(PRACTICE)}

\begin{center}
	\begin{circuitikz}[scale=0.75]
		\draw 
		(0,0) to[I, l=$I_\text{S}$, invert] ++(0,3)
		(0,0) to[open, i=$I_\text{element}$, invert] ++(0,3);
	\end{circuitikz}
\end{center}
}
\ans{
\begin{center}
	\begin{circuitikz}[scale=0.75]
		\draw 
		(0,0) to[I, l=$I_\text{S}$, invert] ++(0,3)
		(0,0) to[open] ++(.5,0)
			to[open, v_=$V_\text{element}$] ++(0,3)
		(0,0) to[open, i=$I_\text{element}$, invert] ++(0,3);
	\end{circuitikz}
\end{center}
}

\qitem\label{ques:resistor_current}{\textbf{(PRACTICE)}

\begin{center}
	\begin{circuitikz}[scale=0.75]
		\draw (0,0) to[R, i=$I_\text{element}$] ++(-4,0);
	\end{circuitikz}
\end{center}
}
\ans{
\begin{center}
	\begin{circuitikz}[scale=0.75]
		\draw (0,0) to[R, v=$V_\text{element}$, i=$I_\text{element}$] ++(-4,0);
	\end{circuitikz}
\end{center}
}

\qitem\label{ques:resistor_voltage}{\textbf{(PRACTICE)}

\begin{center}
	\begin{circuitikz}[scale=0.75]
		\draw (0,0) to[R, v=$V_\text{element}$] ++(-4,0);
	\end{circuitikz}
\end{center}
}
\ans{
\begin{center}
	\begin{circuitikz}[scale=0.75]
		\draw (0,0) to[R, v=$V_\text{element}$, i=$I_\text{element}$] ++(-4,0);
	\end{circuitikz}
\end{center}
}
\end{enumerate}