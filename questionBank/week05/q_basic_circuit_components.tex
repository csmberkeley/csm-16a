% Author: Mudit Gupta, Yannan Tuo
% Email: mudit+csm16a@berkeley.edu, ytuo@berkeley.edu
% Edited Spring 2019 by Lydia Lee, lydia.lee@berkeley.edu
\qns{Basic Circuit Components}

\sol{
Description: Introduction to basic circuit components.
Mentors: do a mini lecture on what is charge, what is voltage, and what is current. See lecture notes for definitions. 
}

In this problem, we will introduce the fundamental circuit components. 

\begin{enumerate}

\qitem{
What is a voltage source?
}

\ans{
Firstly, a voltage source is represented in this manner: 
\begin{center}
    \begin{circuitikz}
        \draw(0,0)
	    to[V=$V$] ++(1,0);
    \end{circuitikz}
\end{center}

Essentially, a voltage source \textbf{guarantees} that the potential at its positive end will be $V$ more than the potential at its negative end, no matter what. 
}

\qitem{
 What is a current source?  
}

\ans{
A current source is represented in this manner: 
\begin{center}
    \begin{circuitikz}
        \draw(0,0)
	     to[I, l=$I$] ++(1,0);
    \end{circuitikz}
\end{center}
A current source \textbf{guarantees} that the current passing through the unit in the direction of the arrow will be its designated value. 
}

\qitem{What is voltage? What is a voltage drop?}

\ans{
For our discussion, it suffices to think of voltage as a kind of driver for current. Current is the movement of charges. A voltage difference forces current to move from the point (node) that has higher voltage, to the point that has lower voltage. 

Voltage drop is the voltage lost (decline of nodal voltage) across a circuit component. 
}

\qitem{What happens in this case if $V_1 \neq V_2$?

\begin{center}
    \begin{circuitikz}
    \draw(0,4)
	%to[short] ++(0,-1)
	to[V_=$V_1$] ++(0,-4)
	to[short] node[ground] {} ++(0,-1);
	
	\draw(0,4)
	to[short] ++(2,0)
	to[V_=$V_2$] ++(0,-4)
	to[short] ++(-2,0);
    \end{circuitikz}
\end{center}

}

\ans{
Let us designate the potential at the positive end of $V_1$ to be $V^+_1$, the potential at the negative end of $V_1$ to be $V^-_1$, the potential at the positive end of $V_2$ to be $V^+_2$, and the potential at the negative end of $V_2$ to be $V^-_2$. $V^-_1$ and $V^-_2$ are equal to 0 because of the ground. Then, the potential across $V_1$ is $V^+_1$, and the potential across $V_2$ is $V^+_2$. Since $V^+_1$ and $V^+_2$ are connected by a wire, they must be the same voltage; we know that a wire does not affect a circuit's behavior, so the voltage must stay constant across it. 
This means that $V^+_1 = V^+_2$. However, we know that the voltage potential $V^+_1 - V^-_1$ is not equal to $V^+_2 - V^-_1$ as given in the question. Hence, we see that we cannot have two voltage sources connected in this configuration.
}

\qitem {What happens in this case if $I_1 \neq I_2$?
\begin{center}
    \begin{circuitikz}
    \draw(0,0) 
    to[I=$I_1$] ++(0,2)
    to[I=$I_2$] ++(0,2);
    \end{circuitikz}
\end{center}
}

\ans{
The current source at the bottom guarantees that through that wire there will be $I_1$ current going through, and the current source at the top guaranteed that $I_2$ current goes through that wire. This is a contradiction, and is not theoretically possible in a circuit. 

Also, look at the point in between the two current sources. $I_1$ enters on one end, and $I_2$ leaves on the other end. This is impossible.
}

\qitem{
What is a resistor? 
}

\ans{
A resistor is represented in this manner: 
\begin{center}
	\begin{circuitikz}

	\draw(0,0)
	to[R, v=$ $, i=$ $] ++(3,0);
	\end{circuitikz}
	\end{center}
A resistor is a circuit unit designed to 'resist' the flow of current. Following convention, there is a "voltage drop" across a resistor from the positive end to the negative end. The voltage drop across a resistor is $V_R = I_R R$, where $V_R$ is the voltage drop, $I_R$ is the current through the resistor and $R$ is the resistance of the resistor. 
}

\qitem{What is power?}

\ans{
Power is the rate at which work is done, where work is in terms of electrical energy.

For circuits, the power \textit{consumed} or \textit{dissipated} by a device is $P = IV$, where the current and voltage abide by passive sign convention.
}

\qitem{What is the passive sign convention?}

\sol{Mentors please note that is a walkthrough question. Students will most likely not know anything about this, so in that case, don't bother giving them time to work on it by themselves}

\ans{
The passive sign convention as we deal with it means that given some arbitrary circuit element (it doesn't matter what it is), we use the following convention:
\begin{center}
	\begin{circuitikz}
		\ctikzset{resistor = european}
		\draw (0,0) to[R, v=$V_\text{element}$, i=$I_\text{element}$] ++(4,0);
	\end{circuitikz}
\end{center}
We can think of this in two ways:
\begin{itemize}
	\item Given a voltage $V_\text{element}$, we define the current $I_\text{element}$ as entering through the node marked $+$ with respect to $V_\text{element}$.
	\item Given a current $I_\text{element}$, we define the voltage $V_\text{element}$ such that $I_\text{element}$ enters through the node marked $+$ with respect to $V_\text{element}$.
\end{itemize}

We can then calculate the power dissipated/consumed by the element with the following (which is always true, regardless of what type of component it is):
$$P_\text{element} = I_\text{element}V_\text{element}$$

Note that this is \textit{dissipated} power. In other words, calculating a positive power value means the component is consuming power, and if we calculate the power and it comes out negative, the component is generating power.

\textbf{THIS NEEDS TO BE HEAVILY REVISED AND BEEFED UP}

\textbf{Common Misconceptions:}
\begin{itemize}
	\item Active components do not necessarily dissipate negative power! Consider the following circuit:
	\begin{center}
		\begin{circuitikz}
			\draw
			(0,0) node[ground] () {}
				to[V=$1\si{\volt}$, invert] ++(0,2)
				to[R=$1\si{\ohm}$] ++(2,0)
			(2,0) node[ground] () {}
				to[V=$2\si{\volt}$, invert] ++(0,2);
		\end{circuitikz}
	\end{center}
	When calculating the power dissipated by the LHS voltage source, we see the current flows counterclockwise about the circuit. With passive sign convention, we calculate the power $P_{V_1} = 1\si{\volt} \times 1\si{\ampere}$, which is positive! The left-side voltage source is dissipating power.

	\item Passive sign convention assumes a circuit is what we call a ``black box'' element. That is, we draw a box around it and pretend like we don't know what's going on inside. That said, you can choose $V_\text{element}$ of a voltage source to be in the opposite direction of the source's given voltage.
\end{itemize}


% In order to be consistent with this, we have to be careful when analyzing circuits in the following manner: when trying to figure out the direction of the current through a resistor and the voltage drop across it, mark one end of the resistor to the 'positive' end and the other end to the 'negative' end. You can do this arbitrarily. 

% Then, we define the direction of the current to be from the positive end to the negative end. 

% We define the voltage drop across the resistor to be $V^+ - V^-$, where $V^+$ is the voltage of the node we marked negative, and $V^-$ is the voltage that we marked negative. 

% One consequence of this convention is that one of two things will happen, (1) both the voltage drop and the current through a resistor will end up having negative values (which ensures that power = $VI$ is positive). What this case means is that the direction we assumed for the current was incorrect, and that current actually goes in the opposite direction!
% (2) both the voltage drop and the current through a resistor will end up being positive, and power will still be positive. 

% Also, a final note is that to maintain this convention, for voltage sources and current sources, we mark current going from the negative end to the positive end. 

}





\end{enumerate}