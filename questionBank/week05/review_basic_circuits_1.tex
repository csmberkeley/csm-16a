\sol{
\qns{Circuit Elements Part 1}

It may help to draw the $I-V$ curves for these devices.

\begin{center}
	\begin{tabular}{|>{\centering\arraybackslash}m{3cm}|>{\centering\arraybackslash}m{3cm}|>{\centering\arraybackslash}m{3cm}|>{\centering\arraybackslash}m{3cm}|}
		\hline
		Name & Voltage Source & Current Source & Resistor\\\hline
		Symbol & 
			\begin{circuitikz}
				\draw (0,0) to[V=$V_{S}$, invert] ++(0,2);
			\end{circuitikz}
			&
			\begin{circuitikz}
				\draw (0,0) to[I, l=$I_S$] ++(0,2);
			\end{circuitikz}
			&
			\begin{circuitikz}
				\draw (0,0) to[R=$R$, v=$V_R$, i=$I_R$] ++(0,2);
			\end{circuitikz}
			\\\hline
		Relevant Information & \begin{tabular}{@{}c@{}}Constant Voltage \\ Undefined Current\end{tabular} & \begin{tabular}{@{}c@{}}Constant Current \\ Undefined Voltage\end{tabular} & $V_R = I_RR$ \\\hline
	\end{tabular}
\end{center}

\textbf{Common Misconceptions:}
\begin{itemize}
	\item Forgetting passive sign convention.
	\item Trying to use Ohm's Law ($V_R = I_R R$) for components that aren't resistors.
	\begin{itemize}
		\item Out of Scope: Other passive components have linear relationships between their voltage and current, they simply have complex components (we call the imaginary portion reactance and the real component resistance---the full complex value is called impedance)
		\item Out of Scope: Ohm's Law is generally interpreted as a differential (i.e. $R = \frac{dV_R}{dI_R}$), meaning it deals with the \textit{slope} of the $I-V$ curve rather than the true value. This is something they'll encounter (and need) in 16B and beyond.
	\end{itemize}
\end{itemize}
}