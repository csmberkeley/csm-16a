% Author: Mudit Gupta, Elena Herbold
% Email: mudit+csm16a@berkeley.edu
%       eherbold@berkeley.edu

\qns{Solutions of linear equations}

\sol{Prereq: Students should be comfortable solving a three-variable system of equations using GE with the forward/backward elimination method. Additionally, they should know how to convert a solution with a free variable from equations describing the solution set into vector notation. \\
Description: Simple mechanical gaussian elimination problem + some insight about free variables}


\begin{enumerate}
\qitem Consider the following set of linear equations:

$$2x + 3y +5z = 0$$
$$-1x -4y -10z = 0$$
$$x-2y-8z = 0$$

Place these equations into a matrix, and row reduce the matrix.

\sol{
  Note to mentors: When you do Gaussian Elimination -- start by making $a_{2,1}$ = 0 using some multiple of $a_{1, 1}$. Next, make $a_{3, 1}$ = 0 using some multiple of $a_{1, 1}$. Next, make $a_{3, 2}$ = 0 by using some multiple of $a_{2, 2}$. In this last step, when you use row 2's pivot to subtract out row 3, the first element of row 3 will not be affected (it will remain 0). This is because in the previous steps, we got rid of the first element of row 2 as well. This is what I like to call the zig zag method of doing Gaussian Elimination. (Elena: I call this the 'staircase', and I this is the Gaussian Elimination method 16A currently teaches. I think it is officially called forward/backward elimination.) Start at the top left, move down the column. Then start again at the top of the second column and move down. 
}

\ans{
  $$\begin{bmatrix} 
  2 & 3 & 5 \\
  -1 & -4 & -10 \\
  1 & -2 & -8 
  \end{bmatrix}$$

  $$R_2 = R_2 + \frac{1}{2}R_1$$
  $$R_3 = R_3 - \frac{1}{2}R_1$$

  $$\begin{bmatrix} 
  2 & 3 & 5 \\
  0 & -2.5 & -7.5 \\
  0 & -3.5 & -10.5 
  \end{bmatrix} \\$$

  Make the numbers nicer by dividing row 2 by -2.5, and multiplying row 3 by -2. This is always a good thing to do if you realize your numbers are getting messy! (Also, feel free to keep all the numbers as non-fractional values by finding the least common multiple of the two numbers you are trying to cancel out.) 

  $$R_2 = \frac{1}{-2.5}R_2$$ 
  $$R_3 = -2R_3$$

  $$\begin{bmatrix} 
  2 & 3 & 5 \\
  0 & 1 & 3 \\
  0 & 7 & 21 
  \end{bmatrix} $$

  $$R_3 = R_3 - 7R_2$$

  $$\begin{bmatrix} 
  2 & 3 & 5 \\
  0 & 1 & 3 \\
  0 & 0 & 0 
  \end{bmatrix}$$

}

\qitem
Convert the row reduced matrix back into equation form. 

\ans{
  $$\begin{bmatrix} 2 & 3 & 5 \\0 & 1 & 3 \\0 & 0 & 0 \end{bmatrix} \begin{bmatrix} x \\ y \\ z \end{bmatrix} = \begin{bmatrix} 0 \\ 0 \\ 0 \end{bmatrix}$$
  $$2x + 3y + 5z = 0$$
  $$0x + 1y + 3z = 0$$
  $$ 0x + 0y + 0z = 0$$
}

\qitem
Intuitively, what does the last equation from the previous part tell us?

\ans{
  It tells us that there are infinite solutions to the equations. $0x + 0y + 0z = 0$ is satisfied by $\textbf{any}$ $x, y, z$. 
}

\sol {
  If students are confused at this point about why we can infer this, their confusion is well justified. Suppose that there were \textbf{4} equations in 3 variables -- 3 of them were linearly independent, and the fourth one was $0x + 0y + 0z = 0$, then the system still has just 1 solution. The last equation is never \textit{used} in some sense. Feel free to talk about this with students. Present it as: what if you had 4 equations, you wrote them in matrix form, got pivots in all rows except for one where you got a row of all 0s -- are there still infinite solutions? The answer is no. 
}

\qitem {Now that we've established that this system has infinite solutions, does every possible combination of $x, y, z \in \mathbb{R}$ solve these equations}

\sol{This is supposed to be a quick part}

\ans{No. $x=1, y=1, z=1$ doesn't work, for instance.}

\qitem
What is the general form (in the form of a constant vector multiplied by a variable $t$) of the infinite solutions to the system? 

\sol{
  Explain why $z$ is the free variable. (Because it is the one that doesn't have a pivot in the corresponding column). Also explain what "general form" means if students are confused.
}

\ans{
  $z$ is a free variable. If $z = t$, then
  $$y = -3z = -3t$$
  $$2x + 3y + 5z = 0 \implies 2x -9t +5t = 0 \implies 2x = 4t \implies x = 2t$$

  The general solution is then $t \begin{bmatrix} 2 \\ -3 \\ 1 \end{bmatrix}$. What this means is that any multiple of the vector $\begin{bmatrix} 2 \\ -3 \\ 1 \end{bmatrix}$ will satisfy the equations. Try it!
}


\end{enumerate}