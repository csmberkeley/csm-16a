% Daniel Aranki <daranki@cs.berkeley.edu>
% Alan Dong <alan_dong@berkeley.edu>
% Kimmy Ko <ko.kimmy@berkeley.edu>

% Henry Sun <tigerhs1998@berkeley.edu> Reformatted for CSM

\newcommand{\stateRankTransMatrix}{\ensuremath{\begin{bmatrix}
\frac{7}{10} & \frac{1}{10} & \frac{1}{10} \\
\frac{1}{10} & \frac{6}{10} & \frac{1}{10} \\
\frac{2}{10} & \frac{3}{10} & \frac{8}{10}
\end{bmatrix}}}

\qns{StateRank Car Rentals (21 points)}

You are an analyst at StateRank Car Rentals, which operates in California, Oregon, and Nevada.
You are hired to analyze the number of rental cars going into and out of each of the three states (CA, OR, and NV).
The number of cars in each state on day $n \in \{0,1,\ldots\}$ can be represented by the state vector $\vec{s}[n] = \begin{bmatrix} s_\text{CA}[n] \\ s_\text{OR}[n] \\ s_\text{NV}[n] \end{bmatrix}$.
The state vector follows the state evolution equation $\vec{s}[n+1] = \mathbf{A}\vec{s}[n], \forall n \in \{0,1,\ldots\}$, where the transition matrix, $\mathbf{A}$, of this linear dynamic system is
\[
\mathbf{A} = 
\stateRankTransMatrix.
\]

\sol {Depending on the familiarity level of the students with flow matrices, the instructor may wish to first draw out the flow diagram. This is also a good time to go over definitions like column-stochastic matrix; but most importantly, there should be an overview of how eigenvalues and eigenvectors control the stability, or the end result, of the system. \\
\begin{center}
\begin{tikzpicture}[>=stealth]

    \tikzset{main node/.style={circle,draw,minimum size=1.5cm,inner sep=0pt,align=center}}

    \node[main node] (ca) {CA};

    \node[main node] (or) [below left = 3cm and 3cm of ca]  {OR};

    \node[main node] (nv) [below right = 3cm and 3cm of ca] {NV};



    \path[->,draw,thick]

	(ca) edge [loop above, ultra thick] node [above,sloped] {\fbox{$\frac{7}{10}$}} (ca)

    (ca) edge [bend right=45, ultra thick] node [left=7pt] {\fbox{$\frac{1}{10}$}} (or)

    (ca) edge [bend left=45, ultra thick] node [right=7pt] {\fbox{$\frac{2}{10}$}} (nv)

	(or) edge [bend right=25, ultra thick] node [above left=-7pt] {\fbox{$\frac{1}{10}$}} (ca)

	(or) edge [loop below, ultra thick] node [below,sloped] {\fbox{$\frac{6}{10}$}} (or)

	(or) edge [bend left=10, ultra thick] node [below,sloped] {\fbox{$\frac{3}{10}$}} (nv)

	(nv) edge [bend left=25, ultra thick] node [above right=-7pt] {\fbox{$\frac{1}{10}$}} (ca)

	(nv) edge [bend left=30, ultra thick] node [below,sloped] {\fbox{$\frac{1}{10}$}} (or)

	(nv) edge [loop below, ultra thick] node [below,sloped] {\fbox{$\frac{8}{10}$}} (nv);

\end{tikzpicture}
\end{center}
}

\begin{enumerate}

\qitem {We denote the eigenvalue/eigenvector pairs of the matrix $\mathbf{A}$ by
\[
\left(
\lambda_1=1,
\vec{u}_1 = \begin{bmatrix}
50\\40\\110
\end{bmatrix}
\right)
,
\left(
\lambda_2,
\vec{u}_2 = \begin{bmatrix}
0\\-10\\10
\end{bmatrix}
\right)
, \text{ and }
\left(
\lambda_3,
\vec{u}_3 = \begin{bmatrix}
-10\\0\\10
\end{bmatrix}
\right).
\]

\textbf{Find the eigenvalues $\lambda_2$ and $\lambda_3$ corresponding to the eigenvectors $\vec{u}_2$ and $\vec{u}_3$, respectively.}
Note that since $\lambda_1=1$ is given, you don't have to calculate it.
}

\ans{

Recall that if $\vec{u},\lambda$ are an eigenpair of a matrix $\mathbf{A}$, then $\mathbf{A} \vec{u} = \lambda \vec{u}$.
By left-multiplying the eigenvectors by $\mathbf{A}$, we get:
\begin{enumerate}
\item $
\stateRankTransMatrix
\begin{bmatrix}
0\\-10\\10
\end{bmatrix} = \begin{bmatrix}
0\\-5\\5
\end{bmatrix} = 0.5 \cdot \begin{bmatrix}
0\\-10\\10
\end{bmatrix}$, which means that $\lambda_2 = 0.5 = \frac{1}{2}$.
\item $
\stateRankTransMatrix
\begin{bmatrix}
-10\\0\\10
\end{bmatrix} = \begin{bmatrix}
-6\\0\\6
\end{bmatrix} = 0.6 \cdot \begin{bmatrix}
-10\\0\\10
\end{bmatrix}$, which means that $\lambda_3 = 0.6 = \frac{3}{5}$.
\end{enumerate}
}

\qitem {
Suppose that the initial number of rental cars in each state on day $0$ is
\[\vec{s}[0] = \begin{bmatrix}
7000\\5000\\8000
\end{bmatrix}
=
100 \vec{u}_1 - 100 \vec{u}_2 - 200 \vec{u}_3,
\]
where $\vec{u}_1, \vec{u}_2$ and $\vec{u}_3$ are the eigenvectors from part (a).

After a very large number of days $n$, how many rental cars will there be in each state?\\
\textbf{That is, i) calculate
\[
\vec{s}^* = \lim_{n\to\infty} \vec{s}[n]
\]
\underline{and} ii) show that the system will indeed converge to $\vec{s}^*$ as $n\rightarrow \infty$ if it starts from $\vec{s}[0]$.}

\textit{Hint:} If you didn't solve part (a), the eigenvalues satisfy $\lambda_1=1, |\lambda_2|<1$ and $|\lambda_3|<1$.
}

\ans{

We know that $\vec{s}[n] = \mathbf{A}^n \vec{s}[0]$.
We also know that $\vec{s}[0] = 100 \vec{u}_1 - 100 \vec{u}_2 -200 \vec{u}_3$.

Therefore, we can write:
\begin{align*}
\vec{s}[n] &= \mathbf{A}^n \vec{s}[0] \\
&= \mathbf{A}^n \left( 100 \vec{u}_1 - 100 \vec{u}_2 -200 \vec{u}_3 \right) \\
&= 100 \mathbf{A}^n \vec{u}_1 - 100 \mathbf{A}^n \vec{u}_2 - 200 \mathbf{A}^n \vec{u}_3 \\
&= 100 \lambda_1^n \vec{u}_1 - 100 \lambda_2^n \vec{u}_2 - 200 \lambda_3^n \vec{u}_3
\end{align*}

From that, we can write:
\[\lim_{n\to\infty} \vec{s}[n] = \lim_{n\to\infty} 100 \lambda_1^n \vec{u}_1 - 100 \lambda_2^n \vec{u}_2 - 200 \lambda_3^n \vec{u}_3\]

Since $|\lambda_2| < 1$ and $|\lambda_3| < 1$, we know that:
\[\lim_{n\to\infty} \left(- 100 \lambda_2^n \vec{u}_2 - 200 \lambda_3^n \vec{u}_3\right) = \vec{0}\]

From that and the fact that $\lambda_1 = 1$, we are left with:
\begin{align*}
\lim_{n\to\infty} \vec{s}[n] &= \lim_{n\to\infty} 100 \lambda_1^n \vec{u}_1 \\
&= \lim_{n\to\infty} 100 \cdot 1^n \vec{u}_1 \\
&= 100 \vec{u}_1 \\
&= \begin{bmatrix}
5000\\4000\\11000
\end{bmatrix}
\end{align*}
}

\end{enumerate}
