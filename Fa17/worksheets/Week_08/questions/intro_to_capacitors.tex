% Author: Yannan Tuo, Emily Gosti, Anwar Baroudi
% Email: ytuo@berkeley.edu, egosti@berkeley.edu

\qns{Introduction to Capacitors}

\sol{
\\Description: Introduction to capacitors.
\\Prerequisite: Understanding of basic circuit components, voltage, current, resistors.
\\Mentors: See lecture notes for definition supplements
}

In this problem, we will introduce the fundamental circuit components. 

\begin{enumerate}

\qitem{
What is a capacitor?
}

\ans{
Firstly, a capacitor is represented in this manner: 
\begin{center} 
	\begin{circuitikz}[line width=1pt]
	\draw(0,0)
		to[C, l_=$C$, v<=$ $] ++(0,1);
	\end{circuitikz}
	\end{center}

Physically speaking, it is two conductive plates separated by some (usually non-conductive) material. The parameters of interest here are:\\
$A$: the area of the surface of the plates that are facing each other.\\
$\epsilon$: the permitivity of the material between the plates. \\
$d$: the distance between the two plates.
}

\qitem{
What is capacitance?
}

\ans{
Capacitance (C) is the measure of how much charge (Q) can be stored for some given amount of voltage (V). It is often written as:
\[C=\frac{Q}{V} \text{ or } Q = CV\]
It can also be written as:
$$C = \frac{\epsilon A}{d}$$
where $\epsilon$ is the "permitivity" of the material between the plates of the capacitor, $A$ is the area of the surface of the plates facing each other, and $d$ is the distance between the two plates.
}

\qitem{What happens to the following circuit (specifically with respect to the voltage across and current through the capacitor) when
\\a) the capacitor becomes fully charged?
\\b) after the capacitor is fully charged, we remove the voltage source and replace it with a short?}

\begin{center} 
    \begin{circuitikz}
	\draw(0,4)
	to[V_, l=$V$] ++(0,-2)
	to[short] node[ground] {} ++(0,-1);
	
	\draw(0,4)
	to[short] ++ (2, 0)
	to[C, l^=$C$, v>=$ $, invert] ++(0, -2)
	to[R, l = $R$] ++(-2, 0);
	

	\end{circuitikz}
	\end{center}
\ans{
When the capacitor is fully charged, then there is no current flowing through the capacitor; $i = 0$.
\begin{center} 
    \begin{circuitikz}
	\draw(0,4)
	to[V_, l=$V$] ++(0,-2)
	to[short] node[ground] {} ++(0,-1);
	
	\draw(0,4)
	to[short] ++ (2, 0)
	to[C, l^=$C$, v>=$ $, i = $i$] ++(0, -2)
	to[R, l = $R$] ++(-2, 0);
	

	\end{circuitikz}
	\end{center}

If we replace the voltage source with a short, then the capacitor starts discharging because of the voltage difference across the resistor.

\begin{center} 
    \begin{circuitikz}
	\draw(0,4)
	to[short, i = $i$] ++(0,-2)
	to[short] node[ground] {} ++(0,-1);
	
	\draw(0,4)
	to[short] ++ (2, 0)
	to[C, l^=$C$, v>=$ $] ++(0, -2)
	to[R, l = $R$] ++(-2, 0);
	

	\end{circuitikz}
	\end{center}
}

\qitem{Using physical properties of capacitors, derive the equation to find the equivalent capacitance for two capacitors in parallel, $C_1$ and $C_2$ in the figure below. Use this equation to find a relationship between the total charge of the two capacitors and the charge of each individual capacitor.

\begin{center} 
    \begin{circuitikz}
	\draw(0,2)
	to[C, l_=$C_1$, v>=$ $] ++(0,-2);
	
	
	\draw(0,2)
	to[short] ++(2,0)
	to[C, l_=$C_2$, v>=$ $] ++(0, -2)
	to[short] ++(-2,0);
	
	
	\draw(1,2)
	to[short, *-o] ++(0,1);
	
	\draw(1, 0)
	to[short, *-o] ++(0, -1);
	
	

	\end{circuitikz}
	\end{center}
}
\ans{
Recall the formula $C = \frac{\epsilon A}{d}$. Let $C_1$ have an area of $A_1$, and let $C_2$ have an area of $A_2$. We can think of these capacitors as having a combined area of $A_1 + A_2$, so $$C_{eq} = \frac{\epsilon (A_1+A_2)}{d} = \frac{\epsilon A_1}{d} + \frac{\epsilon A_2}{d} = C_1 + C_2$$
Now, since $Q = CV$, the total charge $Q_{tot} = (C_1 + C_2)V$. We can leave $V$ as is because the voltage is the same across both capacitors, since they share the same nodes. Then,
$$Q_{tot} = (C_1 + C_2)V = C_1V + C_2V = Q_1 + Q_2$$
}

\qitem {Using physical properties of capacitors, derive the equation for the equivalent capacitance of two capacitors in series.
\begin{center} 
    \begin{circuitikz}
	\draw(0,4)
	to[C, l_=$C_1$, v>=$ $] ++(0, -2)
	to[C, l_=$C_2$, v>=$ $] ++(0, -2);
	\end{circuitikz}
	\end{center}
}
\ans{
Let $C_1$ have a distance $d_1$ between its plates, and let $C_2$ have a distance $d_2$ between its plates. We can picture $C_1$ and $C_2$ as one capacitor with a combined distance $d_1 + d_2$. Then, assuming the areas for both capacitors are both $A$,  we have $C_{eq} = \frac{\epsilon A}{d_1 + d_2}$. To make the math simpler, we invert both sides:
$$\frac{1}{C_{eq}} = \frac{d_1 + d_2}{\epsilon A} = \frac{d_1}{\epsilon A} + \frac{d_2}{\epsilon A} = \frac{1}{C_1} + \frac{1}{C_2}$$
Inverting it back, we get:
$$C_{eq} = \frac{1}{\frac{1}{C_1} + \frac{1}{C_2}} = \frac{C_1C_2}{C_1 + C_2}$$
}
\meta{Note: the equations for combining capacitors is the opposite of the equations for combining resistors. }


\end{enumerate}