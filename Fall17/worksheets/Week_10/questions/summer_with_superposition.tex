% Author: Emily Gosti, Yannan Tuo, Sukrit Arora
% Email: egosti@berkeley.edu, ytuo@berkeley.edu, sukrit.arora@berkeley.edu

\qns{Voltage Summers}

\meta{
Description: Show how to construct a voltage summer and give intuition on how to use it.
}


\begin{enumerate}

\qitem{
%https://www.allaboutcircuits.com/textbook/semiconductors/chpt-8/averager-summer-circuits/
Calculate $V_{out}$ in terms of $V_1$ and $V_2$. \\

 \begin{center}
    \begin{circuitikz} 
    \draw (0,0) 
    node[op amp, yscale=-1] (opamp) {}
    (opamp.+) node[left] {}
    to[short] ++(-2, 0)
    to[R = $R_1$] ++(-2,0)
    to[V_ = $V_1$] ++(-2,0)
    to[short] node[ground] {} ++(0,0)
    (opamp.-) node[left] {}
    to[short] ++(-1,0)
    to[short] ++(0,-2)
    to[short] ++(3.5, 0)
    (opamp.out) node[right] {}
    to[R = $R_4$] ++(0,-3)
    to[R = $R_3$] ++(0, -1)
    to[short] node[ground] {} ++(0,-1)
    (opamp.out) node[right] {}
    to[short, l = $V_o$, *-o] ++(0.5, 0);
    
    \draw(-3, 0.5)
    to[short] ++(0, -2)
    to[R, l^ = $R_2$] ++(-2,0)
    to[V_ = $V_2$] ++(-2, 0)
    to[short] node[ground] {} ++(0,0);
    
    \end{circuitikz}
    \end{center}
}

\meta{

}

\ans{
% To use superposition, we want to consider each of the sources individually. Let's first null out $V_2$ and consider just $V_1$.  Repeating this process again, but this time nulling $V_1$ instead of $V_2$...
Let's look at $V_+$ in this circuit and try to find its value in terms of $V_1$ and $V_2$.
$$V_+ = \frac{\frac{V_1}{R_1}+\frac{V_2}{R_2}}{\frac{1}{R_1}+\frac{1}{R_2}}$$
Assuming $R_1 = R_2$,
$$\frac{V_1 + V_2}{2}$$
This circuit takes the average of the input voltages! So, when we feed in $V_+$ to a non-inverting op-amp, we get
$$V_o = \frac{V_1 + V_2}{2}(1 + \frac{R_4}{R_3})$$
}

\qitem{
What values should we select for $R_1$, $R_2$, $R_3$, and $R_4$ such that $V_{out} = V_1 + V_2$?
}

\ans{
If we choose $R_3$ and $R_4$ such that the gain is 2, then we get $V_o = V_1 + V_2$. One viable set of values is $R_1 = 1$, $R_2 = 1$, $R_3 = 1$, $R_4 = 1$. $R_1$ and $R_2$ have to equal each other, as established in the previous part, but we need to choose $R_3$ and $R_4$ such that (1 + $\frac{R_4}{R_3}$) equals the number of voltages we are summing up. This cancels out the averaging effect of linking up all the voltages to each other.
}

\qitem{
What would $V_{o}$ be if we added another voltage $V_3$ in the same configuration as $V_1$ and $V_2$? If we added $V_4$? How would we change the resistor values?
}

\ans{
% We don't have to calculate $V_o$ using superposition again. We can simply follow the pattern that we established in parts (a) and (b) and add *insert here*. This is called a non-inverting voltage summer, which we can use to add up voltages when we design circuits.
If we added another voltage $V_3$, $V_+$ would be $\frac{V_1 + V_2 + V_3}{3}$. This means, to get $V_o = V_1 + V_2 + V_3$, we have to adjust $R_3$ and $R_4$ such that the gain is 3 instead of 2. One possible such combination is $R_3 = 1$ and $R_4 = 2$.
}

\qitem{
 Alternatively, we could also use an inverting op amp to build an inverting summer. Given $R_1 = R_2 = R_3 = R$, calculate the value of $V_{out}$ given the following circuit:

   \begin{center}
    \begin{circuitikz} 
    \draw (0,0) 
    node[op amp] (opamp) {}
    (opamp.-) node[left] {}
    to[short] ++(-1, 0)
    to[R, l = $R_1$, *-o] ++(-2,0)
    to[short, l = $V_{1}$] ++(0,0) 
    (opamp.+) node[left] {} 
    to[short] node[ground] {} ++(0,-1)
    (opamp.out) node[right] {}
    to[short] ++(0, 1.5)
    to[R, l = $R_3$, i = $i$] ++(-2.4, 0)
    to[short] ++(0,-1)
    (opamp.out) node[right, *-o] {}
    to[short, l = $V_o$, *-o] ++(0.5, 0);
    
    \draw (-2.2, 0.5)
    to[short] ++(0, -2)
    to[R, l = $R_2$, *-o] ++(-2,0)
    to[short, l = $V_{2}$] ++(0,0)
    
    
    \end{circuitikz}
    \end{center}
    
}
\ans{
Let's call the current across $R_1$ $i_1$, and the current across $R_2$ $i_2$. Using KCL at the $V^-$ terminal, we have $i = i_1 + i_2$. The voltage at $V^- = 0$ because of our golden rules. Then, substituting in our values for the currents, we have $$\frac{V_o}{R_3} = \frac{0-V_1}{R_1} + \frac{0-V_2}{R_2}$$
Since $R_1 = R_2 = R_3 = R$, we can simplify this to $$V_o = -V_1 + -V_2$$
\\We see that $V_o$ is just the sum of all the input voltages, except negative. We can apply similar logic as in the previous parts to sum more voltages together.
   \begin{center}
    \begin{circuitikz} 
    \draw (0,0) 
    node[op amp] (opamp) {}
    (opamp.-) node[left] {}
    to[short] ++(-1, 0)
    to[R, l = $R_1$, i = $i_1$] ++(-2,0)
    to[short, *-o] ++(-1,0)
    to[short, l = $V_{1}$] ++(0, 0)
    (opamp.+) node[left] {} 
    to[short] node[ground] {} ++(0,-1)
    (opamp.out) node[right] {}
    to[short] ++(0, 1.5)
    to[R, l = $R_3$, i = $i$] ++(-2.4, 0)
    to[short] ++(0,-1)
    (opamp.out) node[right, *-o] {}
    to[short, l = $V_o$, *-o] ++(0.5, 0);
    
    \draw (-2.2, 0.5)
    to[short] ++(0, -2)
    to[R, l = $R_2$, i = $i_2$] ++(-2,0)
    to[short, *-o] ++(-1,0)
    to[short, l = $V_{2}$] ++(0, 0)
    
    
    \end{circuitikz}
    \end{center}

}

\end{enumerate}